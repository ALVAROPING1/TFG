\documentclass[es]{uc3mthesisIEEE}
\usepackage[es,enableTraceability,enableCaptions]{srs}


\usepackage{import}
\usepackage{enumitem}  % control item separation -> \begin{itemize}[nosep]
\usepackage{lipsum}  % dummy text
\usepackage{placeins}  % \FloatBarrier -> prevents figures and tables from passing that point

\usepackage{mymacros}  % report-specific macros
\usepackage[most]{tcolorbox} % boxes
\usepackage{pdflscape} % landscape
\usepackage[nounderscore]{syntax} % grammar
\usepackage{dirtree}  % directory trees
\usepackage{tikz}
\usepackage{pgfplots}
\pgfplotsset{compat=newest}
\usetikzlibrary{decorations.pathmorphing}
\usepackage{pgfgantt}  % gannt diagrams

\graphicspath{{img/}}

\setlength{\grammarparsep}{8pt}
\setlength{\grammarindent}{10.5em}

% repository URL
\newcommand{\myrepo}{\url{https://github.com/ALVAROPING1/CreatorCompiler}}
\newcommand{\myreportrepo}{\url{https://github.com/ALVAROPING1/TFG}}

% silence ht warnings
\usepackage{silence}
\WarningFilter{latex}{`h' float specifier changed to `ht'}


% REFERENCES
\addbibresource{references.bib}  % bibliography file
\import{}{glossary.tex}  % glossary file


% DOCUMENT

% setup
\degree{Grado en Ingeniería de Software}
\title{Desarrollo de un Compilador Genérico de Lenguaje Ensamblador para el Simulador CREATOR}
\shorttitle{Desarrollo de un Compilador de Ensamblador para CREATOR}
\author{Álvaro Guerrero Espinosa}
\advisors{Félix García Carballeira}
\place{Leganés, Madrid, España}
\date{Junio 2025}

\begin{document}

  % COVER
  \makecover


  % EPIGRAPH
  \makeepigraph
    {Cualquier ingenuo puede escribir código que un computador puede entender. Los buenos programadores escriben código que los humanos pueden entender}  % quote
    {Martin Fowler}  % author
    {Refactoring: Improving the Design of Existing Code}  % source


  % ABSTRACT
  \begin{abstract}
    Este trabajo presenta un ensamblador genérico, robusto y flexible para su
    uso en el simulador CREATOR que permita mejorar las capacidades de esta
    herramienta. Está diseñado para ser utilizado por los alumnos de la
    asignatura de \textit{Estructura de computadores}.

    Este ensamblador es capaz de utilizar multiples arquitecturas distintas
    definidas mediante un fichero de configuración. En esta configuración se
    pueden definir muchas características distintas, como las instrucciones,
    registros, y directivas definidas. Una vez seleccionada una arquitectura,
    este ensamblador permite compilar programas en lenguaje ensamblador para
    dicha arquitectura, utilizando funcionalidades avanzadas como expresiones
    aritméticas o etiquetas como valores numéricos. Además, está diseñado para
    ser capaz de generar mensajes de error útiles que ayuden a los usuarios a
    aprender el lenguaje ensamblador.

    Debido a estas características, con este ensamblador se busca mejorar la
    enseñanza del lenguaje ensamblador a los estudiantes de esta asignatura, y
    mejorar las capacidades de CREATOR para el desarrollo de nuevas arquitecturas.

    \keywords{Ensamblador \sep CREATOR \sep Genérico \sep Compilador}
  \end{abstract}


  % ACKNOWLEDGEMENTS
  \begin{acknowledgements}
    Querría comenzar agradeciendo a mis padres por haberme permitido llegar
    hasta aquí, al pagar la matrícula y permitirme centrarme en los estudios sin
    necesidad de trabajar a tiempo parcial.

    En segundo lugar, quiro agradecer a mi tutor Félix por permitirme trabajar
    en este proyecto y proponerme la idea cuándo no sabía sobre qué hacer este
    trabajo. Este proyecto me ha permitido aplicar muchos de los conocimientos
    aprendidos en la carrera y aprender mucho.

    Quiero darle las gracias a José Antonio por ayudarme con su conocimiento de
    ensamblador a decidir las funcionalidades a implementar en el sistema y a
    encontrar fallos en el mismo. Creó un muy buen caso de prueba para verificar
    el correcto funcionamiento y rendimiento del sistema completo. Me aguantó
    muchas tardes en la universidad mientras trabajaba en este proyecto y me
    permitió discutir con él problemas que estaba teniendo.

    También quiero darle las gracias a Alejandro Calderón por ser uno de mis
    mejores profesores durante la carrera y darme consejos durante la
    realización de este proyecto.

    También me gustaría darle las gracias a todos mis profesores por enseñarme
    los conocimientos necesarios para realizar este proyecto, y a mis amigos de
    la universidad por permitirme llegar hasta aquí y ayudarme cuando lo
    necesitava.

    Por último, me gustaría darle las gracias a la comunidad FOSS. Aunque no he
    interactuado mucho directamente, la mayoría de las herramientas que uso
    fueron creadas por ellos y he aprendido mucho leyendo posts en StackOverflow
    y GitHub. Las pocas veces que he contribuido a algún proyecto FOSS siempre
    he sido bienvenido, y admiro su esfuerzo por mejorar el mundo con
    \textit{software} libre.
  \end{acknowledgements}


  % TOC
  \tableofcontents
  \listoffigures
  \listoftables


  % THESIS
  \begin{thesis}
    \includefrom{parts/}{introduction.tex}
    \includefrom{parts/}{state_of_the_art.tex}
    \includefrom{parts/}{analysis.tex}
    \includefrom{parts/}{design.tex}
    \includefrom{parts/}{implementation.tex}
    \includefrom{parts/}{validation.tex}
    \includefrom{parts/}{planning.tex}
    \includefrom{parts/}{conclusions.tex}
  \end{thesis}


  % BIBLIOGRAPHY
  \cleardoublepage
  \label{bibliography}
  \printbibliography[heading=bibintoc]


  % GLOSSARY
  \cleardoublepage
  \label{glossary}
  \printglossaries
  % \printnoidxglossaries[type=\acronymtype]  % slower, but no need to do $ makeglossaries report


  % APPENDICES
  % \begin{appendices}
  %   \chapter{My stuff}
  %   \lipsum
  % \end{appendices}


\end{document}
