\makeglossaries
% \makenoidxglossaries  % slower, but no need to do $ makeglossaries report.tex


% usage:
% - \gls{term}: regular stuff
% - \Gls{term}: first uppercase
% - \glspl{term}: plural
% - \Glspl{term}: you know the drill
% - \glsdisp{term}{custom text}: link with custom text
% https://www.overleaf.com/learn/latex/Glossaries

% ╔════════════════════════════════════════════════════════════════════════════╗
% ║                                  Glossary                                  ║
% ╚════════════════════════════════════════════════════════════════════════════╝

\newglossaryentry{theorem prover}{
  name = {provador de teoremas},
  plural = {provadores de teoremas},
  description = {Herramienta que permite la generación automática de demostraciones de teoremas}
}

\newglossaryentry{toolchain}{
  name = {toolchain},
  description = {Conjunto de herramientas utilizadas para compilar y desarrollar software},
  see = {compiler}
}

\newglossaryentry{macro}{
  name = {macro},
  description = {Regla o patrón que especifica como una cierta entrada debe ser convertirtida a una cadena de reemplazo}
}

\newglossaryentry{expression}{
  name = {expresión},
  plural = {expresiones},
  description = {Elemento sintáctico de un lenguaje de programación que se puede evaluar para obtener un valor},
  see = {programming language}
}

\newglossaryentry{bigint}{
  name = {bigint},
  description = {Número entero de tamaño arbitrariamente grande}
}

\newglossaryentry{sentence}{
  name = {sentencia},
  description = {Elemento sintáctico de un lenguaje de programación que indica una acción a realizar},
  see = {programming language}
}

\newglossaryentry{program}{
  name = {programa},
  description = {Secuencia de instrucciones escrita en un lenguaje de programación para que un computador la ejecute},
  see = {programming language,computer}
}

\newglossaryentry{programming language}{
  name = {lenguaje de programación},
  plural = {lenguajes de programación},
  description = {Sistema de notación para escribir un programa},
  see = {program}
}

\newglossaryentry{high-level language}{
  name = {lenguaje de alto nivel},
  plural = {lenguajes de alto nivel},
  description = {Lenguaje de programación con fuertes abstracciones de los detalles del computador},
  see = {programming language,computer}
}

\newglossaryentry{low-level language}{
  name = {lenguaje de bajo nivel},
  plural = {lenguajes de bajo nivel},
  description = {Lenguaje de programación con pocas o ninguna abstracción de los detalles del computador y su ISA},
  see = {programming language,computer,ISA}
}

\newglossaryentry{assembly}{
  name = {lenguaje ensamblador},
  plural = {lenguajes ensamblador},
  description = {Lenguaje de programación de bajo nivel con una fuerte correspondencia entre las instrucciones en el lenguaje y las instrucciones soportadas por el computador},
  see = {programming language}
}

\newglossaryentry{assembler}{
  name = {ensamblador},
  plural = {ensambladores},
  description = {Compilador que traduce un programa escrito en lenguaje ensamblador a código máquina},
  see = {compiler,machine code}
}

\newglossaryentry{machine code}{
  name = {código máquina},
  description = {Código de un computador formado por instrucciones utilizadas para controlar su comportamiento, típicamente mezcladas con datos},
  see = {computer}
}

\newglossaryentry{compilation}{
  name = {compilación},
  plural = {compilaciones},
  description = {Proceso de traducción de un programa escrito en un lenguaje de programación a otro},
  see = {programming language}
}

\newglossaryentry{compiler}{
  name = {compilador},
  plural = {compiladores},
  description = {Programa cuyo propósito es compilar otro programa},
  see = {compilation,program}
}

\newglossaryentry{formatter}{
  name = {formateador de código},
  plural = {formateadores de código},
  description = {Herramienta que aplica un formato específico a un código fuente, típicamente modificando espacios, saltos de línea, e indentación}
}

\newglossaryentry{linter}{
  name = {linter},
  description = {Herramienta que realiza un análisis de un código fuente para buscar errores y código sospechoso}
}

\newglossaryentry{compile-time}{
  name = {tiempo de compilación},
  plural = {compiladores},
  description = {Tiempo durante el cual un programa se compila},
  see = {compilation}
}

\newglossaryentry{computer}{
  name = {computador},
  plural = {computadores},
  description = {Máquina que puede ser programada para llevar a cabo una secuencia de operaciones}
}

\newglossaryentry{directive}{
  name = {directiva de ensamblador},
  plural = {directivas de ensamblador},
  description = {Instrucción del ensamblador para realizar una tarea o cambiar un ajuste},
  see = {assembler}
}

\newglossaryentry{data directive}{
  name = {directiva de datos},
  plural = {directivas de datos},
  description = {Directiva de ensamblador que especifica datos a añadir en la memoria de datos, con su tipo, tamaño, y alineamiento},
  see = {directive,data memory}
}

\newglossaryentry{memory}{
  name = {memoria},
  description = {Dispositivo de almacenamiento utilizado para almacenar información}
}

\newglossaryentry{data memory}{
  name = {memoria de datos},
  description = {Segmento de la memoria de un computador utilizado para almacenar los datos de un programa},
  see = {memory,computer}
}

\newglossaryentry{text memory}{
  name = {memoria de texto},
  description = {Segmento de la memoria de un computador utilizado para almacenar las instrucciones de un programa},
  see = {memory,computer,instruction}
}

\newglossaryentry{instruction}{
  name = {instrucción},
  plural = {instrucciones},
  description = {Orden dada al procesador de un computador para que realize alguna acción},
  see = {processor,computer}
}

\newglossaryentry{pseudo-instruction}{
  name = {pseudo-instrucción},
  plural = {pseudo-instrucciones},
  description = {Instrucción permitida en un ensamblador que se reemplaza por una secuencia de instrucciones durante la compilación},
  see = {instruction,assembler}
}

\newglossaryentry{processor}{
  name = {procesador},
  plural = {procesadores},
  description = {Componente de un computador encargado de ejecutar instrucciones y realizar operaciones},
  see = {instruction,computer}
}

\newglossaryentry{null byte}{
  name = {byte nulo},
  description = {Byte con el valor 0}
}

\newglossaryentry{word}{
  name = {palabra},
  description = {Tamaño de un registro de un computador},
  see = {computer,register}
}

\newglossaryentry{register}{
  name = {registro},
  description = {Localización de acceso rápido para un procesador con una pequeña cantidad de memoria},
  see = {processor}
}

\newglossaryentry{immediate}{
  name = {valor inmediato},
  plural = {valores inmediatos},
  description = {Argumento de una instrucción cuyo valor se encuentra codificado en la propia instrucción, en vez de en un registro o memoria. Típicamente números entero},
  see = {instruction,register,memory}
}

\newglossaryentry{edit-distance}{
  name = {distancia de edición},
  description = {Métrica de cadenas de caracteres que mide la diferencia entre dos secuencias, definida como la cantidad mínima de modificaciones requeridas para transformar una cadena en la otra. Típicamente las operaciones permitidas son inserción, sustitución, o borrado de un caracter, y se puede añadir la transposición de dos caracteres adyacentes.},
}

\newglossaryentry{type system}{
  name = {sistema de tipos},
  description = {Sistema lógico con un conjunto de reglas que asignan un tipo a cada elemento del lenguaje, típicamente variables, expresiones, y funciones}
}

\newglossaryentry{dynamic typing}{
  name = {sistema de tipos dinámico},
  description = {Sistema de tipos en el cual una variable puede tomar valores de diferentes tipos y la comprobación de tipos debe hacerse durante la ejecución del programa},
  see = {type system}
}

\newglossaryentry{static typing}{
  name = {sistema de tipos estático},
  description = {Sistema de tipos en el cual el tipo de una variable se decide durante la compilación, y esta solo puede tomar valores del tipo correspondiente},
  see = {type system,compilation}
}

\newglossaryentry{weak typing}{
  name = {sistema de tipos débil},
  description = {Sistema de tipos con reglas poco estrictas y muchas conversiones de tipos implícitas},
  see = {type system}
}

\newglossaryentry{strong typing}{
  name = {sistema de tipos fuerte},
  description = {Sistema de tipos con reglas estrictas y pocas conversiones de tipos implícitas, típicamente solo entre diferentes tipos de números},
  see = {type system}
}

\newglossaryentry{memory safety}{
  name = {seguridad de memoria},
  description = {Protección contra errores y vulnerabilidades de seguridad causadas por errores en el uso y gestión de la memoria},
  see = {memory}
}

\newglossaryentry{library}{
  name = {biblioteca},
  description = {Conjunto de recursos que implementan una funcionalidad y cuya finalidad es ser utilizado por otro programa},
  see = {program}
}

\newglossaryentry{binding}{
  name = {binding},
  description = {API que provee código diseñado para permitir que un lenguaje de programación utilice una biblioteca escrita en otro lenguaje},
  see = {API,programming language}
}

\newglossaryentry{internal node}{
  name = {nodo interno},
  plural = {nodos internos},
  description = {Nodo de un árbol que tiene hijos}
}

\newglossaryentry{leaf node}{
  name = {nodo hoja},
  plural = {nodos hoja},
  description = {Nodo de un árbol que no tiene hijos}
}

\newglossaryentry{parser}{
  name = {parser},
  description = {Componente de un compilador encargado de realizar el análisis sintáctico del código},
  see = {compiler}
}

\newglossaryentry{lexer}{
  name = {lexer},
  description = {Componente de un compilador encargado de transformar el código a una secuencia de tokens},
  see = {compiler,token}
}

\newglossaryentry{symbol table}{
  name = {tabla de símbolos},
  description = {Estructura de datos utilizada por un compilador para almacenar la información asociada con cada identificador},
  see = {compiler}
}

\newglossaryentry{grammar}{
  name = {gramática},
  description = {Descripción formal de la sintaxis de un lenguaje, basada en el uso de reglas de transformación de secuencias de símbolos},
  see = {compiler}
}

\newglossaryentry{ambiguous grammar}{
  name = {gramática ambigua},
  plural = {gramáticas ambiguas},
  description = {Gramática para la que existe alguna cadena que se puede obtener con varios árboles de sintaxis distintos},
  see = {grammar}
}

\newglossaryentry{operator-precedende grammar}{
  name = {gramática de precedencia de operadores},
  plural = {gramáticas de precedencia de operadores},
  description = {Gramática en la que ninguna producción tiene una parte derecha vacía o con varios símbolos no terminales juntos},
  see = {grammar}
}

\newglossaryentry{forward-reference}{
  name = {referencia hacia delante},
  plural = {referencias hacia delante},
  description = {Uso de un elemento, típicamente una función o variable, antes de su definición en el código de un programa}
}

\newglossaryentry{token}{
  name = {token},
  description = {Secuencia de caracteres un significado conjunto tratada como una unidad}
}

\newglossaryentry{renderer}{
  name = {renderizador},
  description = {Componente encargado de generar una imagen a partir de datos de entrada}
}

\newglossaryentry{regex}{
  name = {expresión regular},
  plural = {expresiones regulares},
  description = {Secuencia de caracteres que especifica un patrón de texto}
}

% ╔════════════════════════════════════════════════════════════════════════════╗
% ║                                  Acronyms                                  ║
% ╚════════════════════════════════════════════════════════════════════════════╝

\newglossaryentrywithacronym{ISA}{Instruction Set Architecture}
{Modelo abstracto de la arquitectura de un procesador, incluyendo el conjunto de instrucciones que puede ejecutar}

\newglossaryentrywithacronym{AST}{Abstract Syntax Tree}
{Estructura de datos utilizada para representar la estructura de un programa}

\newglossaryentrywithacronym{API}{Application Programming Interface}
{Interfaz de un programa diseñada para ser utilizada por otro programa}

\newglossaryentrywithacronym{FOSS}{Free and Open Source Software}
{Software disponible bajo una licencia que concede el derecho de utilizar, modificar, y distribuir el software a cualquier persona libre de cargos}

\newglossaryentrywithacronym{CLI}{Command-line Interface}
{Interfaz que permite la interacción con un programa mediante líneas de texto}

\newacronym{json}{JSON}{JavaScript Object Notation}
\newacronym{HTML}{HTML}{Hypertext Markup language}
\newacronym{CSS}{CSS}{Cascading Style Sheets}
\newacronym{JS}{JS}{JavaScript}
\newacronym{TS}{TS}{TypeScript}
\newacronym{wasm}{Wasm}{WebAssembly}
\newacronym{FSF}{FSF}{Free Software Foundation}
\newacronym{OSI}{OSI}{Open Source Initiative}
\newacronym{LGPL}{LGPL}{GNU Lesser General Public License}
\newacronym{GPL}{GPL}{GNU General Public License}
\newacronym{GAS}{GAS}{GNU Assembler}
