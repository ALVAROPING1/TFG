\makeglossaries
% \makenoidxglossaries  % slower, but no need to do $ makeglossaries report.tex


% usage:
% - \gls{term}: regular stuff
% - \Gls{term}: first uppercase
% - \glspl{term}: plural
% - \Glspl{term}: you know the drill
% - \glsdisp{term}{custom text}: link with custom text
% https://www.overleaf.com/learn/latex/Glossaries

% ╔════════════════════════════════════════════════════════════════════════════╗
% ║                                  Glossary                                  ║
% ╚════════════════════════════════════════════════════════════════════════════╝

\newglossaryentry{expression} {
  name = {expresión},
  plural = {expresiones},
  description = {Elemento sintáctico de un lenguaje de programación que se puede evaluar para obtener un valor},
  see = {programming language}
}

\newglossaryentry{sentence} {
  name = {sentencia},
  description = {Elemento sintáctico de un lenguaje de programación que indica una acción a realizar},
  see = {programming language}
}

\newglossaryentry{program} {
  name = {programa},
  description = {Secuencia de instrucciones escrita en un lenguaje de programación para que un computador la ejecute},
  see = {programming language,computer}
}

\newglossaryentry{programming language} {
  name = {lenguaje de programación},
  plural = {lenguajes de programación},
  description = {Sistema de notación para escribir un programa},
  see = {program}
}

\newglossaryentry{high-level language} {
  name = {lenguaje de alto nivel},
  plural = {lenguajes de alto nivel},
  description = {Lenguaje de programación con fuertes abstracciones de los detalles del computador},
  see = {programming language,computer}
}

\newglossaryentry{low-level language} {
  name = {lenguaje de bajo nivel},
  plural = {lenguajes de bajo nivel},
  description = {Lenguaje de programación con pocas o ninguna abstracción de los detalles del computador y su ISA},
  see = {programming language,computer,ISA}
}

\newglossaryentry{assembly} {
  name = {lenguaje ensamblador},
  description = {Lenguaje de programación de bajo nivel con una fuerte correspondencia entre las instrucciones en el lenguaje y las instrucciones soportadas por el computador},
  see = {programming language}
}

\newglossaryentry{assembler} {
  name = {ensamblador},
  plural = {ensambladores},
  description = {Compilador que traduce un programa escrito en lenguaje ensamblador a código máquina},
  see = {compiler,machine code}
}

\newglossaryentry{machine code} {
  name = {código máquina},
  description = {Código de un computador formado por instrucciones utilizadas para controlar su comportamiento, típicamente mezcladas con datos},
  see = {computer}
}

\newglossaryentry{compilation} {
  name = {compilación},
  plural = {compilaciones},
  description = {Proceso de traducción de un programa escrito en un lenguaje de programación a otro},
  see = {programming language}
}

\newglossaryentry{compiler} {
  name = {compilador},
  plural = {compiladores},
  description = {Programa cuyo propósito es compilar otro programa},
  see = {compilation,program}
}

\newglossaryentry{computer} {
  name = {computador},
  plural = {computadores},
  description = {Máquina que puede ser programada para llevar a cabo una secuencia de operaciones},
}

\newglossaryentry{directive} {
  name = {directiva de ensamblador},
  plural = {directivas de ensamblador},
  description = {Instrucción del ensamblador para realizar una tarea o cambiar un ajuste},
  see = {assembler}
}

\newglossaryentry{data directive} {
  name = {directiva de datos},
  plural = {directivas de datos},
  description = {Directiva de ensamblador que especifica datos a añadir en la memoria de datos, con su tipo, tamaño, y alineamiento},
  see = {directive,data memory}
}

\newglossaryentry{memory} {
  name = {memoria},
  description = {Dispositivo de almacenamiento utilizado para almacenar información},
}

\newglossaryentry{data memory} {
  name = {memoria de datos},
  description = {Segmento de la memoria de un computador utilizado para almacenar datos de un programa},
  see = {memory,computer}
}

\newglossaryentry{text memory} {
  name = {memoria de texto},
  description = {Segmento de la memoria de un computador utilizado para almacenar las instrucciones de un programa},
  see = {memory,computer,instruction}
}

\newglossaryentry{instruction} {
  name = {instrucción},
  plural = {instrucciones},
  description = {Orden dada al procesador de un computador para que realize alguna acción},
  see = {processor,computer}
}

\newglossaryentry{pseudo-instruction} {
  name = {pseudo-instrucción},
  plural = {pseudo-instrucciones},
  description = {Instrucción permitida en un ensamblador que se reemplaza por una secuencia de instrucciones durante la compilación},
  see = {instruction,assembler}
}

\newglossaryentry{processor} {
  name = {procesador},
  plural = {procesadores},
  description = {Componente de un computador encargado de ejecutar instrucciones y realizar operaciones},
  see = {instruction,computer}
}

% ╔════════════════════════════════════════════════════════════════════════════╗
% ║                                  Acronyms                                  ║
% ╚════════════════════════════════════════════════════════════════════════════╝

\newglossaryentrywithacronym{ISA}
{Instruction Set Architecture}
{Modelo abstracto de la arquitectura de un procesador, incluyendo el conjunto de instrucciones que puede ejecutar}

\newglossaryentrywithacronym{AST}
{Abstract Syntax Tree}
{Estructura de datos utilizada para representar la estructura de un programa}

\newglossaryentrywithacronym{API}
{Application Programming Interface}
{Interfaz de un programa diseñada para ser utilizada por otro programa}

\newglossaryentrywithacronym{FOSS}
{Free and Open Source Software}
{Software disponible bajo una licencia que concede el derecho de utilizar, modificar, y distribuir el software a cualquier persona libre de cargos}

\newacronym{json}{JSON}{JavaScript Object Notation}

\newacronym{js}{JS}{JavaScript}
