\chapter{Plan del proyecto}\label{chap:planning}

\section{Planificación}
\subsection{Metodología}

\section{Presupuesto}

Esta sección explicará el presupuesto del proyecto, basado en la planificación
de tiempo descrita en la anterior sección. Primero, en la \subsectionref{costs},
se detallará el coste del proyecto, y, tras esto, en la \subsectionref{offer}, se
expondrá la oferta presentada al cliente.

\subsection{Coste del proyecto}\label{subsec:costs}

La \tableref{project-info} contiene un resumen de las características del
proyecto y el presupuesto total.

\makeatletter

\begin{table}[htb]
  \ttabbox[\FBwidth]
    {\caption{Información del proyecto}\label{tab:project-info}}
    {
      \begin{tabular}{>{\bfseries}p{3.5cm}p{9cm}}
        \toprule
        Título            & \textit{\@title} \\ \midrule
        Autor             & \@author \\ \midrule
        Departamento      & Departamento de Informática \\ \midrule
        Fecha de inicio   & 15 de julio del 2024 \\ \midrule
        Fecha de fin      & 16 de junio del 2025 \\ \midrule
        Duración          & 11 meses \\ \midrule
        Presupuesto total & 0 \euro \\ % TODO:
        \bottomrule
      \end{tabular}
    }
\end{table}

\makeatother

Los costes se dividen en costes directos (asociados con el personal y
equipamiento) e indirectos (con una influencia indirecta en el proyecto). Estos
costes no incluirán impuestos, ya que esos se incluyen en
\subsubsectionref{costs-summary}.

\subsubsection{Costes directos}

Los costes directos son los costes relacionados directamente con el desarrollo
del proyecto. Se pueden dividir en costes de personal, que dependen de la
cualificación, experiencia, y localización de cada miembro, y costes de
equipamiento, asociados a las herramientas utilizadas durante el desarrollo.

\noindent
Los costes de personal se pueden dividir en cuatro roles:

\begin{itemize}
    \item \textbf{Jefe de proyecto:} gestiona la planificación del proyecto, y aporta
    \textit{feedback} sobre el desarrollo.
    \item \textbf{Analista:} analiza los requisitos de usuario, realiza la
    arquitectura del sistema, y escribe la documentación.
    \item \textbf{Programador:} implementa las funcionalidades del sistema.
    \item \textbf{Tester:} diseña y realiza las pruebas de las funcionalidades
    del sistema.
\end{itemize}

El tutor realizó el rol de gestor del proyecto, mientras que el estudiante
realizó los otros roles. La \tableref{costs-person} muestra los costes del
personal totales y para cada rol.

\begin{table}[htb]
  \ttabbox[\FBwidth]
    {\caption{Costes de personal}\label{tab:costs-person}}
    {
      \begin{tabular}{lrrr}
        \toprule
        \textbf{Rol} & \textbf{Horas} & \textbf{Coste por hora (\euro/h)} & \textbf{Total (\euro)} \\
        \midrule
        Jefe de proyecto &  60 h & 60,00 & 3.600,00 \\
        Analista         & 220 h & 45,00 & 9.900,00 \\
        Programador      & 250 h & 35,00 & 8.750,00 \\
        Tester           & 130 h & 30,00 & 3.900,00 \\
        \midrule
        \textbf{Total}   & 660 h &       & \textbf{26.150,00 \euro} \\
        \bottomrule
      \end{tabular}
    }
\end{table}

Los costes de equipamiento están asociados a la compra y uso del equipamiento,
incluyendo \textit{software} y \textit{hardware}. Con respecto al
\textit{software}, todas las herramientas utilizadas fueron \gls{FOSS} y, por lo
tanto, no tienen coste asociado. El coste de cada equipo de \textit{hardware}
se calcula teniendo en cuenta el tiempo que este se utiliza para el proyecto con
la siguiente fórmula:

\begin{equation}\label{eq:chargeable-cost}
    c = \frac{C \cdot t \cdot p}{a}
\end{equation}

\noindent
Donde:

\begin{itemize}
    \item $c$: coste amortizado
    \item $C$: coste del equipamiento
    \item $t$: tiempo durante el que se utiliza el equipamiento para el proyecto
    \item $p$: porcentaje del tiempo total que se utilizó para el proyecto
    \item $a$: tiempo de amortización
\end{itemize}

\noindent
La \tableref{costs-equipment} muestra el coste de cada equipamiento y el coste total
de equipamiento.


\begin{table}[htb]
  \ttabbox[\FBwidth]
    {\caption{Costes de equipamiento}\label{tab:costs-equipment}}
    {
      \begin{adjustbox}{max width=\textwidth}
        \begin{tabular}{lrrrrr}
          \toprule
          \textbf{Objeto}   & \textbf{Coste ($C$)} & \textbf{Uso ($p$)} & \textbf{Dedicación ($t$)} & \textbf{Amortización ($D$)} & \textbf{Coste amortizado ($c$)} \\
          \midrule
          PC sobremesa      &  999,00 \euro        & 30 \%              & 11 meses                  &  60 meses                   &  54,95 \euro \\
          Portátil          &  749,00 \euro        & 60 \%              & 11 meses                  &  60 meses                   &  82,39 \euro \\
          Monitor           &  149,99 \euro        & 40 \%              & 11 meses                  &  48 meses                   &  13,75 \euro \\
          Ratón             &   39,99 \euro        & 30 \%              & 11 meses                  &  36 meses                   &   3,67 \euro \\
          Cable HDMI        &    4,99 \euro        & 40 \%              & 11 meses                  &  48 meses                   &   0,46 \euro \\
          \textit{Software} &    0,00 \euro        & 60 \%              & 11 meses                  & 120 meses                   &   0,00 \euro \\
          \midrule
          \textbf{Total}    & 1.942,97 \euro       &                    &                           &                             & \textbf{155,22 \euro} \\
          \bottomrule
        \end{tabular}
      \end{adjustbox}
    }
\end{table}

\subsubsection{Costes indirectos}

Los costes indirectos son aquellos que tienen una influencia indirecta en el
proyecto y no pueden ser asignados a un producto específico, como el consumo
eléctrico, la conexión a internet, o el transporte.

Para el consumo el eléctrico, se asume que el portátil consume $65$ W de media,
el PC de sobremesa $300$ W, y el monitor y ratón $20$ W. Además, de las $660$
horas del proyecto, se ha utilizado el portátil el $70 \%$ del tiempo, y el
monitor, ratón, y PC de sobremesa el $30 \%$ restante. Teniendo este en cuenta,
la energía total utilizada es de $(65 W \cdot 0.7 + 320 \cdot 0.3) \cdot 660 h =
93.390 Wh$.

La conexión a internet es un plan de 1 gbps de fibra óptica, con un precio
de $65,00$ \euro/mes. Esta conexión se comparte por $4$ personas y solo una de
ellas participa en el proyecto, por lo que el coste para el proyecto es un
cuarto de eso.

\noindent
La tabla \tableref{costs-indirect} contiene los costes indirectos totales del
proyecto.

\begin{table}[htb]
  \ttabbox[\FBwidth]
    {\caption{Costes indirectos}\label{tab:costs-indirect}}
    {
      \begin{tabular}{lrrr}
        \toprule
        \textbf{Recurso} & \textbf{Coste unitario} & \textbf{Unidades} & \textbf{Total} \\
        \midrule
        Electricidad &   0.15 \euro/kWh & 93.390 Wh &  14,01 \euro \\
        Internet     &  16,25 \euro/mes & 11 months & 178,75 \euro \\
        Transporte   &      8 \euro/mes & 11 months &  88,00 \euro \\
        \midrule
        \textbf{Total} & & & \textbf{280,76 \euro} \\
        \bottomrule
      \end{tabular}
    }
\end{table}

\subsubsection{Resumen de costes}\label{subsubsec:costs-summary}

\noindent
La \tableref{costs-summary} contiene un resumen de los costes del proyecto,
incluyendo los costes directos e indirectos.

\begin{table}[htb]
  \ttabbox[\FBwidth]
    {\caption{Resumen de costes}\label{tab:costs-summary}}
    {
      \begin{tabular}{lr}
        \toprule
        Personal          & 26.150,00 \euro \\
        Equipamiento      &    155,22 \euro \\
        Costes indirectos &    280,76 \euro \\
        \midrule
        \textbf{Total}    & \textbf{26.585,98 \euro} \\
        \bottomrule
      \end{tabular}
    }
\end{table}

\subsection{Oferta del proyecto}\label{subsec:offer}

La \tableref{project-offer} detalla la oferta del proyecto. En esta oferta se
incluyen unos riesgos esperados del $20 \%$, unos beneficios del $15 \%$, y los
impuestos del IVA, que es del $21 \%$ en España. Teniendo en cuenta esto, el
coste final del proyecto es de \textbf{41.474,14 \euro~(Cuarenta y un mil
cuatrocientos setenta y cuatro euros con catorce centimos)}

\begin{table}[htb]
  \ttabbox[\FBwidth]
    {\caption{Oferta del proyecto}\label{tab:project-offer}}
    {
      \begin{tabular}{lrrr}
        \toprule
        \textbf{Concepto} & \textbf{Incremento (\%)} & \textbf{Valor parcial (\euro)} & \textbf{Coste agregado (\euro)} \\
        \midrule
        Coste del proyecto & -- & 26.585,98 \euro & 26.585,98 \euro \\
        Riesgos            & 20 &  5.317,20 \euro & 31.903,18 \euro \\
        Beneficios         & 15 &  3.987,90 \euro & 35.891,08 \euro \\
        IVA                & 21 &  5.583,06 \euro & 41.474,14 \euro \\
        \midrule
        \textbf{Total}     & 56 &                 & \textbf{41.474,14 \euro} \\
        \bottomrule
      \end{tabular}
    }
\end{table}

\section{Marco regulatorio}

En esta sección se expondrán las restricciones legales que se aplican al
proyecto, y está dividido en tres subsecciones. La primera
(\subsectionref{legislation}), explicará las leyes y regulaciones aplicables al
proyecto. La segunda (\subsectionref{standards}), hará un resumen de los
estándares técnicos utilizados en el proyecto. Por último, la tercera
(\subsectionref{licenses}), indicará las licencias bajo las que se distribuye el
sistema y sus componentes.

\subsection{Legislación}\label{subsec:legislation}

El sistema se ejecuta localmente, y, aunque se puede utilizar en un entorno web,
se ejecuta exclusivamente en el navegador cliente. Además, el sistema no utiliza
ni transmite datos personales, por lo que no se aplica ninguna ley de protección
de datos como la Ley Orgánica de Protección de Datos Personales y Garantía de
los Derechos Digitales (en adelante, ``LOPDGDD''). Aunque el código
\glsdisp{assembly}{ensamblador} que procesa el sistema se podría considerar
propiedad intelectual, este no se almacena ni se envía, por lo que no se aplica
ninguna regulación. Además, no hay ningún riesgo involucrado en la ejecución del
sistema, por lo que no hay ninguna otra regulación que se aplique al sistema.

\subsection{Estándares técnicos}\label{subsec:standards}

\noindent
El sistema utiliza usa los siguientes estándares técnicos:

\begin{itemize}
    \item ISO/IEC 21778:2017 \parencite{JSONStandard}, el estándar del formato
    JSON utilizado para la definición de las \glspl{ISA}.
    \item ISO/IEC 10646:2020 \parencite{UTF-8}, el estándar de la codificación
    UTF-8 usada en las cadenas de caracteres.
    \item La especificación de \gls{wasm} \parencite{wasm-spec}, utilizado para
    ejecutar el compilador en un entorno web.
\end{itemize}

\subsection{Licencias}\label{subsec:licenses}

El sistema utiliza y redistribuye múltiples bibliotecas externas:

\begin{itemize}
    \item \libref{serde}, \libref{serde\string_json}, \libref{ariadne},
    \libref{chumsky}, \libref{regex}, \libref{once\string_cell}
    \libref{num-bigint}, \libref{num-traits}, \libref{wasm-bindgen}, y
    \libref{ansi-to-html} utilizan la licencia MIT \parencite{MIT}, que permite
    utilizar, copiar, modificar, publicar, y redistribuir el código sin
    restricciones.
    \item \libref{self\string_cell} utiliza la licencia Apache 2.0
    \parencite{apache2}, que permite utilizar, copiar, modificar, publicar, y redistribuir el código sin
    restricciones.
\end{itemize}

El sistema desarrollado utiliza la licencia GNU Lesser General Public License
version 3 \parencite{lgpl}, ya que permite el uso, modificación, y
redistribución del código de forma libre con la única restricción de que las
modificaciones se tienen que publicar, asegurando que el sistema continua siendo
abierto.

\section{Entorno socioeconómico}
