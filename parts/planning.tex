\chapter{Plan del proyecto}\label{chap:planning}

Este capítulo describirá el plan del proyecto a desarrollar y está
dividido en cuatro secciones. La primera (\sectionref{plan}), explicará la
planificación del proyecto. La segunda (\sectionref{budget}), describirá el
presupuesto con el que se desarrollará el sistema y sus costes. La tercera
(\sectionref{regulation}), explicará las restricciones legales que se aplican al
proyecto. Por último, la cuarta (\sectionref{environment}), describirá el
entorno socioeconómico en el que se desarrollará el proyecto.

\newcommand{\componentref}[1]{\hyperref[req:#1]{`#1'}}

\section{Planificación}\label{sec:plan}

Esta sección explicará la planificación del proyecto, incluyendo la metodología
utilizada, las etapas en las que se dividirá el proyecto, y la duración de cada
etapa.

\subsection{Metodología}

Actualmente, existen muchas metodologías para el desarrollo de \textit{software}.
Algunas de estas son:

\begin{itemize}
    \item \textbf{Modelo en cascada \parencite{waterfall-model}:} esta
    metodología está fundamentada en la realización de las diversas tareas del
    desarrollo de \textit{software} (análisis, diseño, implementación, pruebas,
    y evaluación) de forma secuencial. Es una buena opción para sistemas
    pequeños y bien definidos, pero al aumentar la complejidad del sistema
    pierde su efectividad, ya que los errores no se descubren hasta el final del
    desarrollo. Además, no permite cambios en los requisitos durante el
    desarrollo, perdiendo la capacidad de reacción a cambios en el entorno y
    necesidades. Debido a esto, en la actualidad, es una metodología poco
    utilizada y no recomendada para sistemas complejos.
    \item \textbf{Modelo en espiral \parencite{spiral-model}:} este es un modelo
    iterativo en el que el desarrollo se fragmenta en varias iteraciones, durante
    las que se realizan prototipos cada vez más complejos que se utilizan
    para orientar el desarrollo. Esto permite explorar los requisitos y
    encontrar problemas más rápidamente, por lo que es una buena opción para
    proyectos complejos en los que los requisitos pueden no estar bien definidos
    inicialmente.
\end{itemize}

Teniendo esto en cuenta, se ha elegido utilizar el modelo en espiral
\parencite{spiral-model}, ya que es más flexible a los cambios y permite
corregir errores en elementos ya desarrollados en anteriores iteraciones.

\subsection{Ciclo de vida}\label{subsec:life-cycle}

El ciclo de vida de esta metodología (representado en la
\figureref{spiral_model}) se puede dividir en cuatro etapas que se realizan en
cada una de las iteraciones del desarrollo. Estas etapas son:

\begin{enumerate}
    \item \textbf{Planificación:} se determinan los requisitos de usuario, que
    se utilizan para decidir los objetivos de la iteración.
    \item \textbf{Análisis:} se analizan los requisitos de usuario para
    identificar los posibles riesgos, y se diseñan las pruebas.
    \item \textbf{Desarrollo y pruebas:} se diseña e implementa el sistema, y se
    realizan las pruebas.
    \item \textbf{Evaluación:} el cliente evalúa el sistema y aporta
    \textit{feedback}. Si este aprueba el sistema, se continúa con la siguiente
    iteración, y si no, se corrigen los problemas antes de continuar.
\end{enumerate}

\svgfigure{spiral_model}{Ciclo de vida del modelo en espiral}

\noindent
El desarrollo del proyecto estará formado por cinco iteraciones:

\begin{enumerate}[label=\Roman*.]
    \item \textbf{Gestor de la arquitectura:} esta etapa se centra en la
    implementación del \componentref{Gestor de la arquitectura}. El objetivo de
    esta etapa es permitir el procesado de la definición de una \gls{ISA} para
    su posterior uso.
    \item \textbf{\Gls{parser}:} esta etapa se centra en la implementación del
    analizador sintáctico. El objetivo de esta etapa es permitir el procesado de
    un código \glsdisp{assembly}{ensamblador}, extrayendo un \gls{AST}.
    \item \textbf{Analizador semántico:} esta etapa se centra en la implementación
    del analizador semántico y su conexión con los componentes anteriores.
    El objetivo de esta etapa es permitir \glsdisp{compilation}{compilar}
    programas, obteniendo un sistema completamente funcional.
    \item \textbf{Integración con CREATOR:} esta etapa se centra en integrar el
    sistema con CREATOR. Su objetivo es permitir el uso del sistema dentro de
    CREATOR.
    \item \textbf{Pulido:} esta etapa se centra en implementar las
    funcionalidades restantes de baja prioridad, que, aunque no son esenciales
    para el funcionamiento del sistema, mejoran su calidad y dan valor añadido.
\end{enumerate}

\subsection{Tiempo estimado}

La planificación de tiempo del proyecto se realizó con un diagrama de Gantt
\parencite{gantt} (\figureref{gantt}). Este diagrama muestra el tiempo dedicado a
cada tarea del desarrollo del sistema, incluyendo todas las iteraciones
realizadas y sus etapas. Las etapas de cada una de las iteraciones son las
descritas en la \subsectionref{life-cycle}, con una etapa adicional de
documentación durante la que se realizará este documento. Además, se ha añadido
una tarea adicional (``Memoria'') para terminar este documento.

El proyecto tuvo una duración final de $12$ meses, con una media de $60$ horas
mensuales (sin contar días no lectivos y vacaciones). Debido a esto, el tiempo
total dedicado al proyecto ha sido de $720$ horas.

\begin{landscape}
  \begin{figure}
    \ffigbox[\FBwidth]
      {\caption{Diagrama de Gantt}\label{fig:gantt}}
      {
        \begin{ganttchart}[
          hgrid,
          time slot format=isodate,
          time slot unit=day,
          x unit=.06cm,
          y unit chart=0.4cm,
          y unit title=.7cm,
          title label font=\footnotesize,
          group label font=\bf\scriptsize,
          bar label font=\scriptsize,
          milestone label font=\it\scriptsize,
          vrule/.style={-},
        ]{2024-06-07}{2025-06-16}
          \gantttitlecalendar{year, month=shortname} \\

          \ganttgroup{I.}{2024-06-07}{2024-06-30}\\
          \ganttbar{Planificación}{2024-06-07}{2024-06-09}\\
          \ganttlinkedbar{Análisis}{2024-06-10}{2024-06-12}\\
          \ganttlinkedbar{Desarrollo y pruebas}{2024-06-13}{2024-06-20}\\
          \ganttlinkedbar{Evaluación}{2024-06-21}{2024-06-24}\\
          \ganttlinkedbar{Documentación}{2024-06-24}{2024-06-30}\\

          \ganttgroup{II.}{2024-07-01}{2024-09-09}\\
          \ganttbar{Planificación}{2024-07-01}{2024-07-03}\\
          \ganttlinkedbar{Análisis}{2024-07-04}{2024-07-17}\\
          \ganttlinkedbar{Desarrollo y pruebas}{2024-07-18}{2024-08-20}\\
          \ganttlinkedbar{Evaluación}{2024-08-21}{2024-08-31}\\
          \ganttlinkedbar{Documentación}{2024-09-01}{2024-09-09}\\

          \ganttgroup{III.}{2024-09-10}{2024-12-20}\\
          \ganttbar{Planificación}{2024-09-10}{2024-09-14}\\
          \ganttlinkedbar{Análisis}{2024-09-15}{2024-09-25}\\
          \ganttlinkedbar{Desarrollo y pruebas}{2024-09-26}{2024-11-30}\\
          \ganttlinkedbar{Evaluación}{2024-12-01}{2024-12-09}\\
          \ganttlinkedbar{Documentación}{2024-12-10}{2024-12-20}\\

          \ganttgroup{IV.}{2025-01-12}{2025-02-21}\\
          \ganttbar{Planificación}{2025-01-12}{2025-01-14}\\
          \ganttlinkedbar{Análisis}{2025-01-15}{2025-01-21}\\
          \ganttlinkedbar{Desarrollo y pruebas}{2025-01-22}{2025-02-08}\\
          \ganttlinkedbar{Evaluación}{2025-02-09}{2025-02-13}\\
          \ganttlinkedbar{Documentación}{2025-02-14}{2025-02-21}\\

          \ganttgroup{V.}{2025-02-22}{2025-04-20}\\
          \ganttbar{Planificación}{2025-02-22}{2025-02-24}\\
          \ganttlinkedbar{Análisis}{2025-02-25}{2025-02-28}\\
          \ganttlinkedbar{Desarrollo y pruebas}{2025-03-01}{2025-03-31}\\
          \ganttlinkedbar{Evaluación}{2025-04-01}{2025-04-10}\\
          \ganttlinkedbar{Documentación}{2025-04-11}{2025-04-18}\\

          \ganttgroup{Memoria}{2025-04-19}{2025-06-14}\\

          \ganttvrule{}{2025-01-01}

          % link iterations
          \ganttlink{elem0}{elem6}
          \ganttlink{elem6}{elem12}
          \ganttlink{elem12}{elem18}
          \ganttlink{elem18}{elem24}
          \ganttlink{elem24}{elem30}
        \end{ganttchart}
      }
  \end{figure}
\end{landscape}

\section{Presupuesto}\label{sec:budget}

Esta sección explicará el presupuesto del proyecto, basado en la planificación
de tiempo descrita en la anterior sección. Primero, en la \subsectionref{costs},
se detallará el coste del proyecto, y, tras esto, en la \subsectionref{offer}, se
expondrá la oferta presentada al cliente.

\subsection{Coste del proyecto}\label{subsec:costs}

La \tableref{project-info} contiene un resumen de las características del
proyecto y el presupuesto total.

\makeatletter

\begin{table}[htb]
  \ttabbox[\FBwidth]
    {\caption{Información del proyecto}\label{tab:project-info}}
    {
      \begin{tabular}{>{\bfseries}p{3.5cm}p{9cm}}
        \toprule
        Título            & \textit{\@title} \\ \midrule
        Autor             & \@author \\ \midrule
        Departamento      & Departamento de Informática \\ \midrule
        Fecha de inicio   & 7 de junio del 2024 \\ \midrule
        Fecha de fin      & 16 de junio del 2025 \\ \midrule
        Duración          & 12 meses \\ \midrule
        Presupuesto total & 48.132,99 \euro \\
        \bottomrule
      \end{tabular}
    }
\end{table}

\makeatother

Los costes se dividen en costes directos (asociados con el personal y
equipamiento) e indirectos (con una influencia indirecta en el proyecto). Estos
costes no incluirán impuestos, ya que esos se incluyen en
\subsubsectionref{costs-summary}.

\subsubsection{Costes directos}

Los costes directos son los costes relacionados directamente con el desarrollo
del proyecto. Se pueden dividir en costes de personal, que dependen de la
cualificación, experiencia, y ubicación de cada miembro, y costes de
equipamiento, asociados a las herramientas utilizadas durante el desarrollo.

\noindent
Los costes de personal se pueden dividir en cuatro roles:

\begin{itemize}
    \item \textbf{Jefe de proyecto:} gestiona la planificación del proyecto, y aporta
    \textit{feedback} sobre el desarrollo.
    \item \textbf{Analista:} analiza los requisitos de usuario, realiza la
    arquitectura del sistema, y escribe la documentación.
    \item \textbf{Programador:} implementa las funcionalidades del sistema.
    \item \textbf{Tester:} diseña y realiza las pruebas de las funcionalidades
    del sistema.
\end{itemize}

El tutor realizó el rol de gestor del proyecto, mientras que el estudiante
realizó los otros roles. La \tableref{costs-person} muestra los costes del
personal totales y para cada rol.

\begin{table}[htb]
  \ttabbox[\FBwidth]
    {\caption{Costes de personal}\label{tab:costs-person}}
    {
      \begin{tabular}{lrrr}
        \toprule
        \textbf{Rol} & \textbf{Horas} & \textbf{Coste por hora (\euro/h)} & \textbf{Total (\euro)} \\
        \midrule
        Jefe de proyecto &  60 h & 60,00 &  3.600,00 \\
        Analista         & 240 h & 45,00 & 10.800,00 \\
        Programador      & 270 h & 35,00 &  9.450,00 \\
        Tester           & 150 h & 30,00 &  4.500,00 \\
        \midrule
        \textbf{Total}   & 720 h &       & \textbf{28.350,00 \euro} \\
        \bottomrule
      \end{tabular}
    }
\end{table}

Los costes de equipamiento están asociados a la compra y uso del equipamiento,
incluyendo \textit{software} y \textit{hardware}. Con respecto al
\textit{software}, todas las herramientas utilizadas fueron \gls{FOSS} y, por lo
tanto, no tienen coste asociado. El coste de cada equipo de \textit{hardware}
se calcula teniendo en cuenta el tiempo que este se utiliza para el proyecto con
la siguiente fórmula:

\begin{equation}\label{eq:chargeable-cost}
    c = \frac{C \cdot t \cdot p}{a}
\end{equation}

\noindent
Donde:

\begin{itemize}
    \item $c$: coste amortizado.
    \item $C$: coste del equipamiento.
    \item $t$: tiempo durante el que se utiliza el equipamiento para el proyecto.
    \item $p$: porcentaje del tiempo total que se utilizó para el proyecto.
    \item $a$: tiempo de amortización.
\end{itemize}

\noindent
La \tableref{costs-equipment} muestra el coste de cada equipamiento y el coste total
de equipamiento.


\begin{table}[htb]
  \ttabbox[\FBwidth]
    {\caption{Costes de equipamiento}\label{tab:costs-equipment}}
    {
      \begin{adjustbox}{max width=\textwidth}
        \begin{tabular}{lrrrrr}
          \toprule
          \textbf{Objeto}   & \textbf{Coste ($C$)} & \textbf{Uso ($p$)} & \textbf{Dedicación ($t$)} & \textbf{Amortización ($a$)} & \textbf{Coste amortizado ($c$)} \\
          \midrule
          PC sobremesa      &  999,00 \euro        & 30 \%              & 12 meses                  &  60 meses                   &  59,94 \euro \\
          Portátil          &  749,00 \euro        & 60 \%              & 12 meses                  &  60 meses                   &  89,88 \euro \\
          Monitor           &  149,99 \euro        & 40 \%              & 12 meses                  &  48 meses                   &  15,00 \euro \\
          Ratón             &   39,99 \euro        & 30 \%              & 12 meses                  &  36 meses                   &   4,00 \euro \\
          Cable HDMI        &    4,99 \euro        & 40 \%              & 12 meses                  &  48 meses                   &   0,50 \euro \\
          \textit{Software} &    0,00 \euro        & 60 \%              & 12 meses                  & 120 meses                   &   0,00 \euro \\
          \midrule
          \textbf{Total}    & 1.942,97 \euro       &                    &                           &                             & \textbf{169,32 \euro} \\
          \bottomrule
        \end{tabular}
      \end{adjustbox}
    }
\end{table}

\subsubsection{Costes indirectos}

Los costes indirectos son aquellos que tienen una influencia indirecta en el
proyecto y no pueden ser asignados a un producto específico, como el consumo
eléctrico, la conexión a internet, o el transporte.

Para el consumo el eléctrico, se asume que el portátil consume $65$ W de media,
el PC de sobremesa $300$ W, y el monitor y ratón $20$ W. Además, de las $720$
horas del proyecto, se ha utilizado el portátil el $70 \%$ del tiempo, y el
monitor, ratón, y PC de sobremesa el $30 \%$ restante. Teniendo este en cuenta,
la energía total utilizada es de $(65 W \cdot 0.7 + 320 \cdot 0.3) \cdot 720 h =
101.880 Wh$.

La conexión a internet es un plan de 1 gbps de fibra óptica, con un precio
de $65,00$ \euro/mes. Esta conexión se comparte por $4$ personas y solo una de
ellas participa en el proyecto, por lo que el coste para el proyecto es un
cuarto de eso.

\noindent
La tabla \tableref{costs-indirect} contiene los costes indirectos totales del
proyecto.

\begin{table}[htb]
  \ttabbox[\FBwidth]
    {\caption{Costes indirectos}\label{tab:costs-indirect}}
    {
      \begin{tabular}{lrrr}
        \toprule
        \textbf{Recurso} & \textbf{Coste unitario} & \textbf{Unidades} & \textbf{Total (\euro)} \\
        \midrule
        Electricidad &   0,15 \euro/kWh & 101.880 Wh &  15,28 \\
        Internet     &  16,25 \euro/mes &  12 months & 195,00 \\
        Transporte   &      8 \euro/mes &  12 months &  96,00 \\
        \midrule
        \textbf{Total} & & & \textbf{306,28 \euro} \\
        \bottomrule
      \end{tabular}
    }
\end{table}

\subsubsection{Resumen de costes}\label{subsubsec:costs-summary}

\noindent
La \tableref{costs-summary} contiene un resumen de los costes del proyecto,
incluyendo los costes directos e indirectos.

\begin{table}[htb]
  \ttabbox[\FBwidth]
    {\caption{Resumen de costes}\label{tab:costs-summary}}
    {
      \begin{tabular}{lr}
        \toprule
        Personal          & 28.350,00 \euro \\
        Equipamiento      &    169,32 \euro \\
        Costes indirectos &    306,28 \euro \\
        \midrule
        \textbf{Total}    & \textbf{28.825,60 \euro} \\
        \bottomrule
      \end{tabular}
    }
\end{table}

\subsection{Oferta del proyecto}\label{subsec:offer}

La \tableref{project-offer} detalla la oferta del proyecto. En esta oferta se
incluyen unos riesgos esperados del $20 \%$, unos beneficios del $15 \%$, y los
impuestos del IVA, que es del $21 \%$ en España. Teniendo en cuenta esto, el
coste final del proyecto es de \textbf{48.132,99 \euro~(Cuarenta y ocho mil
ciento treinta y dos euros con noventa y nueve céntimos)}.

\begin{table}[htb]
  \ttabbox[\FBwidth]
    {\caption{Oferta del proyecto}\label{tab:project-offer}}
    {
      \begin{tabular}{lrrr}
        \toprule
        \textbf{Concepto} & \textbf{Incremento (\%)} & \textbf{Valor parcial (\euro)} & \textbf{Coste agregado (\euro)} \\
        \midrule
        Coste del proyecto & -- & 28.825,60 & 28.825,60 \\
        Riesgos            & 20 &  5.765,12 & 34.590,72 \\
        Beneficios         & 15 &  5.188,61 & 39.779,33 \\
        IVA                & 21 &  8.353,66 & 48.132,99 \\
        \midrule
        \textbf{Total}     & 67 &           & \textbf{48.132,99 \euro} \\
        \bottomrule
      \end{tabular}
    }
\end{table}

\section{Marco regulador}\label{sec:regulation}

En esta sección se expondrán las restricciones legales que se aplican al
proyecto, y está dividido en tres subsecciones. La primera
(\subsectionref{legislation}), explicará las leyes y regulaciones aplicables al
proyecto. La segunda (\subsectionref{standards}), hará un resumen de los
estándares técnicos utilizados en el proyecto. Por último, la tercera
(\subsectionref{licenses}), indicará las licencias bajo las que se distribuye el
sistema y sus componentes.

\subsection{Legislación}\label{subsec:legislation}

El sistema se ejecuta localmente, y, aunque se puede utilizar en un entorno web,
se ejecuta exclusivamente en el navegador cliente. Además, el sistema no utiliza
ni transmite datos personales, por lo que no se aplica ninguna ley de protección
de datos como la Ley Orgánica de Protección de Datos Personales y Garantía de
los Derechos Digitales (en adelante, ``LOPDGDD''). Aunque el código
\glsdisp{assembly}{ensamblador} que procesa el sistema se podría considerar
propiedad intelectual, este no se almacena ni se envía, por lo que no se aplica
ninguna regulación. Además, no hay ningún riesgo involucrado en la ejecución del
sistema, por lo que no hay ninguna otra regulación que se aplique al sistema.

\subsection{Estándares técnicos}\label{subsec:standards}

\noindent
El sistema utiliza usa los siguientes estándares técnicos:

\begin{itemize}
    \item ISO/IEC 21778:2017 \parencite{JSONStandard}, el estándar del formato
    JSON utilizado para la definición de las \glspl{ISA}.
    \item ISO/IEC 10646:2020 \parencite{UTF-8}, el estándar de la codificación
    UTF-8 usada en las cadenas de caracteres.
    \item La especificación de \gls{wasm} \parencite{wasm-spec}, utilizado para
    ejecutar el \gls{compiler} en un entorno web.
\end{itemize}

\subsection{Licencias}\label{subsec:licenses}

El sistema utiliza y redistribuye múltiples bibliotecas externas:

\begin{itemize}
    \item \libref{serde}, \libref{serde\string_json}, \libref{ariadne},
    \libref{chumsky}, \libref{regex}, \libref{once\string_cell}
    \libref{num-bigint}, \libref{num-traits}, \libref{wasm-bindgen}, y
    \libref{ansi-to-html} utilizan la licencia MIT \parencite{MIT}, que permite
    utilizar, copiar, modificar, publicar, y redistribuir el código sin
    restricciones.
    \item \libref{self\string_cell} utiliza la licencia Apache 2.0
    \parencite{apache2}, que permite utilizar, copiar, modificar, publicar, y redistribuir el código sin
    restricciones.
\end{itemize}

El sistema desarrollado utiliza la licencia \gls{LGPL} versión 3
\parencite{lgpl}, ya que permite el uso, modificación, y redistribución del
código de forma libre con la única restricción de que las modificaciones se
tienen que publicar, asegurando que el sistema continúa siendo abierto. Aunque
también se consideró la licencia \gls{GPL} versión 3 \parencite{gpl}, con
características similares, se decidió utilizar la licencia LGPL para ser
compatible con CREATOR, que se distribuye bajo esta misma licencia.

\section{Entorno socioeconómico}\label{sec:environment}

Históricamente, la gran mayoría de los \glspl{computer} de escritorio y
servidores han utilizado las \glsdisp{ISA}{arquitecturas} x86 o x86-64. Estas
\glsdisp{ISA}{arquitecturas} ofrecen un alto rendimiento, pero tienen una mala
eficiencia energética y un alto coste de fabricación debido a su gran
complejidad. Debido a esto, el mercado móvil y de dispositivos del \gls{IoT}
utilizan otras \glsdisp{ISA}{arquitecturas} como ARM o RISC-V que, aunque
ofrecen un peor rendimiento, son mucho más simples, gracias a lo cual pueden
lograr una mejor eficiencia energética y coste de producción.

Recientemente, sin embargo, se ha descubierto que \glsdisp{ISA}{arquitecturas}
simples como ARM pueden lograr un rendimiento similar a x86-64. Además, a
diferencia de x86-64, cuyas licencias para diseñar \glspl{processor} son
difíciles de conseguir por estar controladas por Intel y AMD, obtener una
licencia para ARM es mucho más fácil. Esto es debido a que la venta de estas
licencias es una parte importante del negocio de Arm Holdings, la empresa
propietaria de ARM. La capacidad de diseñar \glspl{processor} propios también
ayuda a reducir los costes de fabricación y operación al eliminar
intermediarios, y ofrece una gran flexibilidad al poder crear \glspl{processor}
para tareas específicas. Debido a esto, recientemente, grandes empresas como
Amazon o Apple han adoptado \glspl{processor} con diseños propios basados en ARM
en muchos de sus productos, incluyendo \glspl{computer} de escritorio y
servidores.

Esto se ve influenciado por la reciente inestabilidad política y económica de
Estados Unidos con políticas como los aranceles. Mientras que Intel y AMD son
empresas americanas y, por lo tanto, se ven afectadas por las políticas de
Estados Unidos, Arm Holdings es una empresa inglesa sin estos problemas.

En este contexto surge RISC-V como una \glsdisp{ISA}{arquitectura} completamente
abierta que no requiere de licencias para el diseño de \glspl{processor}. Esto
reduce mucho más su coste de fabricación al permitir a cualquier empresa diseñar
\glspl{processor} con esta \glsdisp{ISA}{arquitectura}, y elimina muchos
problemas políticos. El diseño modular de RISC-V también fomenta el desarrollo
de extensiones para tareas específicas, que pueden lograr un mayor rendimiento y
eficiencia energética. Sin embargo, su adopción fuera de \glspl{processor} para
\gls{IoT} está siendo muy lenta, posiblemente debido a que al ser una
\glsdisp{ISA}{arquitectura} relativamente reciente, existe poco
\textit{software} y se tiene poca experiencia en su uso y diseño.

Esto lleva a la necesidad de desarrollar buenos simuladores que permitan
desarrollar y aprender a utilizar nuevas \glsdisp{ISA}{arquitecturas} como
RISC-V. Como ya se explicó en el \chapterref{introduction}, el proyecto se ha
integrado en un simulador \gls{FOSS} que contribuye al aprendizaje del
\gls{assembly}. Este simulador se utiliza actualmente en múltiples
universidades entre las que se incluyen la Universidad Carlos III de Madrid, la
Universidad de Castilla-La Mancha, la Universidad de León, la Universidad de
Almería, y la Universidad Sureste del Estado de Missouri. El proyecto
desarrollado ayudará a los estudiantes de estas universidades a aprender el
\gls{assembly}.
