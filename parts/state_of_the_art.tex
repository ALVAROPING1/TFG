\chapter{Estado del arte}\label{chap:state-of-the-art}

En este capítulo se analizará el estado del arte, detallando el estado de las
diferentes tecnologías relacionadas con el desarrollo del proyecto. Este
capítulo estará dividido en cinco secciones. La primera
(\sectionref{simulators}), presentará los simuladores de \gls{assembly}. La
segunda (\sectionref{assemblers}), analizará los \glspl{assembler} utilizados en
la actualidad. La tercera (\sectionref{parser-techniques}), describirá las
diferentes técnicas utilizadas para implementar analizadores sintácticos. Tras
esto, la \sectionref{error-messages}, analizará la información contenida en los
mensajes de error generados por diversos \glspl{compiler}. Por último, la
\sectionref{programming-languages}, estudiará los \glspl{programming language}
utilizados en la actualidad para implementar las diversas herramientas
utilizadas durante el desarrollo de \textit{software}.

% TODO:
% - RISCV/ARM/MIPS
% - WASM/JS
% - NPM/cargo

\section{Simuladores de ensamblador}\label{sec:simulators}

Un simulador de \glsdisp{assembly}{ensamblador} es un \gls{program} que permite
simular la ejecución de un \gls{program} escrito en \glsdisp{assembly}{ensamblador},
simulando el comportamiento de un \gls{processor} que ejecuta el código. Suelen
tener un propósito didáctico, para ayudar al usuario a aprender el
funcionamiento del lenguaje y el \gls{processor}, aunque algunos también se
pueden utilizar de forma profesional para desarrollar una nueva \gls{ISA}.

Estos simuladores están compuestos de dos componentes principales: un
\gls{assembler} y un ejecutor. El \gls{assembler} se encarga de procesar el
código \glsdisp{assembly}{ensamblador} introducido por el usuario a un formato
que pueda ser utilizado por el simulador, mientras que el ejecutor se encarga
de ejecutar el código generado por el \gls{assembler} en el \gls{processor}
simulado. Los simuladores se pueden clasificar en específicos, si estos componentes
solo permiten simular una \gls{ISA} específica, o genéricos, si se pueden
configurar para simular varias \glspl{ISA} distintas.

Como el objetivo del proyecto es la creación de un \gls{assembler} para un
simulador, en esta sección se van a analizar varios simuladores existentes en la
actualidad.

\subsection{Simuladores específicos}

Un simulador específico está diseñado para simular una única \gls{ISA}, como
RISC-V o ARM, y no permite su modificación. Esto hace que sean más simples que
los simuladores genéricos, y, debido a esto, la mayoría de los simuladores en la
actualidad entran dentro de esta categoría. En esta sección se exponen dos
ejemplos de este tipo de simuladores: Spike y Kite

\subsubsection{Spike}

Spike \parencite{spike} es el simulador de referencia para RISC-V, desarrollado
por RISC-V International. Está escrito en C++, y es capaz de emular un sistema RISC-V completo, y está
diseñado para servir como un punto de inicio para ejecutar código RISC-V. Debido
a esto, permite ejecutar casi cualquier \glspl{program}.

Spike cuenta con una interfaz por \glsdisponly{CLI}{línea de comandos (CLI)}
(\figureref{spike}), y permite seleccionar las extensiones de la \gls{ISA}
(contando con soporte para la mayoría de las extensiones estándar) y el modelo
de memoria a utilizar. Permite ejecutar código \gls{instruction} a
\gls{instruction}, visualizar el contenido de los \glspl{register} y la
\gls{memory}, I/O, y ejecutar código en modo usuario y privilegiado. También cuenta
con un modo básico de depuración interactivo, y permite utilizar \progref{gdb}
en caso de necesitar un depurador más avanzado. Aunque es un simulador
específico de RISC-V, permite añadir y probar nuevas \glspl{instruction}.

\svgfigure[0.5]{spike}{Interfaz de Spike}

\subsubsection{Kite}

Kite \parencite{kite} \parencite{kite-gh} es un simulador que modela un
\gls{processor} RISC-V con un \textit{pipeline} de cinco etapas, implementado en
C++. Su primera versión se desarrolló en 2019 con propósitos educativos para la
asignatura de arquitectura de Computadores. Este simulador está basado en el
modelo de \gls{processor} descrito en \textit{Computer Organization and Design,
RISC-V Edition: The Hardware and Software Interface} \parencite{kite-book}. Su
objetivo es ofrecer a los estudiantes un simulador fácil de usar con un modelo
de temporización preciso que siga lo descrito en el libro.

Kite implementa la mayoría de las \glspl{instruction} básicas descritas en el
libro. Este simulador ofrece funcionalidades avanzadas del modelo de
\textit{pipeline}, como comprobaciones de dependencias (riesgos de datos),
\textit{stalls}, memoria caché de datos, y opcionalmente envío adelantado y
predicción de ramas. Además, también cuenta con una opción de depuración que
permite ver el progreso detallado de la ejecución de \glspl{instruction} en el
\textit{pipeline}. Esto ayuda a los estudiantes a mejorar su comprensión del
\textit{pipeline} de un \gls{processor}

Kite cuenta se utiliza mediante una \gls{CLI} (\figureref{kite}) que recibe el
nombre del fichero con el código \glsdisp{assembly}{ensamblador} a ejecutar.
Además, utiliza dos ficheros adicionales, \verb!reg_state! y \verb!mem_state!,
que deben contener el estado inicial de los \glspl{register} y \gls{memory} a
utilizar. El simulador ejecuta el \gls{program} dado, opcionalmente mostrando
información sobre el progreso de la ejecución de \glspl{instruction} en el
\textit{pipeline}. Al terminar la ejecución del \gls{program}, muestra una serie
de estadísticas sobre esta, como el número de ciclos de reloj utilizados, el
número de ciclos perdidos durante \textit{stalls}, los ciclos de reloj
consumidos por cada \gls{instruction} de media, y estadísticas de la memoria
caché. Por último, también muestra el estado final de los \gls{register} y las
direcciones de \gls{memory} accedidas.

\svgfigure[0.5]{kite}{Interfaz de Kite}

\FloatBarrier

\subsection{Simuladores genéricos}

Un simulador genérico permite simular varias \glspl{ISA} distintas. Esto añade
bastante complejidad a su diseño, ya que los componentes necesitan soportar
muchas opciones de configuración distintas para permitir especificar la
\gls{ISA}. En esta sección se exponen dos ejemplos de este tipo de simuladores:
Sail y CREATOR.

\subsubsection{Sail}

Sail \parencite{sail} es un \glsdisp{programming language}{lenguaje} para
definir la \gls{ISA} de un \gls{processor} desarrollado por la Universidad de
Cambridge.

Esta herramienta permite validar la definición de la \gls{ISA} y generar
simuladores en C y OCaml. Además, también permite ejecutar pruebas de validación
y generar versiones de la \gls{ISA} que se pueden ejecutar con un \gls{theorem
prover} para realizar demostraciones sobre el comportamiento de la \gls{ISA}.
Actualmente, está en desarrollo un componente para generar un modelo de
referencia en un lenguaje de descripción de \textit{hardware}. Todas estas
características lo hacen una muy buena opción para el uso profesional, aunque
esto también lo vuelve una mala opción para el uso didáctico debido a su gran
complejidad.

\subsubsection{CREATOR}

CREATOR (didaCtic and geneRic assEmbly progrAmming simulaTOR)
\parencite{CREATOR} es un simulador didáctico desarrollado por el grupo de
investigación ARCOS de la Universidad Carlos III de Madrid. Tiene tanto una
versión web (\figureref{creator}) como una versión de \gls{CLI}, y se desarrolló
para ayudar a los estudiantes a aprender el \gls{assembly} en la asignatura de
\textit{Estructura de Computadores}.

CREATOR cuenta con un editor de código con resaltado de sintaxis para escribir
el código \glsdisp{assembly}{ensamblador}, y una interfaz que permite ver el
estado del \gls{processor} (\glspl{register} de control, enteros y de coma
flotante) y la \gls{memory} (\gls{data memory}, \gls{text memory}, y
\textit{stack}), las \glspl{instruction} cargadas, y diversas estadísticas sobre
la ejecución del \gls{program}. Tiene soporte para la ejecución
\gls{instruction} a \gls{instruction}, I/O, llamadas al sistema,
\textit{breakpoints}, y convención de paso de parámetros.

Con respecto a la modificación de la \gls{ISA}, CREATOR es muy flexible. Permite
modificar la arquitectura del procesador (\glspl{register} definidos y su
nombre, tipo, tamaño y diferentes propiedades), la organización de la
\gls{memory}, y las \glspl{instruction}, \glspl{pseudo-instruction}, y
\gls{directive} permitidas en el código. Con respecto a las \glspl{instruction}
y \glspl{pseudo-instruction}, permite definir su nombre, sintaxis, argumentos y
la codificación de estos en binario, tamaño, número de ciclos para su ejecución,
y su definición, ya sea las acciones a realizar durante su ejecución para las
\glspl{instruction} o la secuencia de \glspl{instruction} por la que
reemplazarla para las \glspl{pseudo-instruction}. Todo esto se realiza mediante
un fichero de configuración en formato \gls{json}, permitiendo el uso de
\gls{JS} para ciertas opciones como la ejecución de una \gls{instruction}.
Utilizando esto, CREATOR cuenta con definiciones para RISC-V y MIPS.

La principal limitación de CREATOR es su \gls{assembler}. Aunque permite
procesar \glspl{program} simples, carece de muchas funcionalidades importantes
como el uso de etiquetas en las \glspl{data directive}. Esto es necesario, por
ejemplo, para la creación del vector de interrupciones utilizado por los
sistemas operativos para gestionar los diversos tipos de interrupciones. Además,
sus mensajes de error son mejorables, y a veces no son capaces de mostrar el
problema real. El sistema a desarrollar se encargará de reemplazar a este
\gls{assembler}, solucionando todos estos problemas.

\graphicfigure[1]{creator}{Interfaz principal de CREATOR (Web)}

\section{Ensambladores}\label{sec:assemblers}

Un \gls{assembler} es un \gls{program} que traduce un \gls{program} escrito en
\gls{assembly} a \gls{machine code} que puede ser ejecutado en un
\gls{processor}. Forma parte de la \gls{toolchain} utilizada para
\glsdisp{compilation}{compilar} un \gls{program}. Aunque la gran mayoría se
diseñan para una \gls{ISA} concreta, algunos pueden procesar código para
múltiples \glspl{ISA} distintas. Cabe destacar que cada \gls{assembler} utiliza
una sintaxis propia, por lo que, en general, el código se escribe para un
\gls{assembler} concreto.

Debido a que el objetivo principal del proyecto es desarrollar un
\gls{assembler}, en esta sección se exponen tres ejemplos de \glspl{assembler}
apliamente utilizados en la actualidad: \gls{GAS}, TCCASM, y NASM.

\subsection{GNU Assembler}

\gls{GAS} \parencite{GNUas} es una familia de \glspl{assembler} desarrollados
por el Proyecto GNU. Están escritos en C y se distribuyen bajo la licencia GNU
GPLv3 \parencite{gpl}. Cada uno de estos \glspl{assembler} está diseñado para
una \gls{ISA} concreta, aunque comparten las características independientes de
la máquina como la sintaxis, la mayoría de \glspl{directive}, y el formato del
fichero binario resultante \parencite{as-manual}. \gls{GAS} soporta más de 50
\glspl{ISA} distintas, entre las que se incluyen RISC-V, MIPS, ARM, y x86. En la
actualidad, \gls{GAS} es uno de los \gls{assembler} más utilizados
\parencite{assembler-usage}.

\gls{GAS} está diseñado para ser utilizado principalmente como parte de la
\gls{toolchain} de GNU, siendo el \textit{backend} por defecto de \progref{gcc},
aunque también es mayormente compatible con otros \glspl{assembler}
\parencite{as-manual}. Debido a esto, sus mensajes de error son mejorables, ya
que obtener un alto rendimiento es más importante y registrar la información
necesaria para generar buenos mensajes de error tiene un cierto coste. Un
ejemplo de esto se puede ver en la figura \figureref{error_gas}.

\svgfigure[0.6]{error_gas}{Mensaje de error del ensamblador de GNU}

\gls{GAS} tiene soporte para muchas funcionalidades avanzadas que lo vuelven un
\gls{assembler} muy flexible. Tiene soporte para etiquetas locales, evaluación
de \gls{expression} aritméticas, \glspl{bigint}, cadenas en UTF-8
\parencite{UTF-8}, \gls{compilation} condicional, emisión de mensajes de error,
definición de símbolos con valores arbitrarios, y \glspl{macro} variadicas
(utilizando una cantidad variable de argumentos) y recursivas, entre otros.

\subsection{TCCASM}

TCCASM es el \gls{assembler} de TCC (Tiny C Compiler) \parencite{tcc}. TCC es un
\gls{compiler} de C diseñado para utilizar poca memoria y ser muy rápido. Es
significativamente más rápido \glsdisp{compilation}{compilando} que
\progref{gcc}, aunque también produce ficheros ejecutables más lentos
\parencite{tcc-speed}. Está escrito en C y se distribuye bajo la licencia GNU
LGPL. TCC soporta principalmente x86 y x86-64, aunque también tiene soporte para
ARM y RISC-V \parencite{tcc-manual} \parencite{tcc-arm} \parencite{tcc-riscv}.

TCC utiliza TCCASM únicamente para procesar ficheros en
\glsdisp{assembly}{ensamblador} y segmentos en \glsdisp{assembly}{ensamblador}
dentro de código C \parencite{tcc-manual}. Gracias a esto, TCC permite
desactivar TCCASM para obtener un ejecutable de TCC más pequeño.

En cuanto al código \glsdisp{assembly}{ensamblador}, TCCASM utiliza una sintaxis
similar a \gls{GAS}. Sin embargo, debido a su objetivo de utilizar poca memoria,
soporta muchas menos funcionalidades que este. En particular, soporta únicamente
enteros de 32 bits, muchas menos \glspl{directive}, y no soporta
\gls{compilation} condicional, definición de símbolos con valores arbitrarios,
ni \glspl{macro}. Además, sus \glspl{expression} aritméticas soportan menos
operadores que \gls{GAS}.

\subsection{NASM}

NASM (Netwide Assembler) \parencite{NASM} es un \gls{assembler} multiplataforma
de x86 y x86-64 diseñado para ser portable y modular. Está escrito en C, y se
distribuye bajo la licencia BSD 2-Clause \parencite{bsd-2c}. Tiene soporte para
todas las extensiones de x86 actualmente conocidas, y permite utilizar muchos
formatos distintos para el fichero binario resultante. Junto con \gls{GAS}, es
uno de los \glspl{assembler} más utilizados en Linux
\parencite{assembler-usage}.

Con respecto al código \glsdisp{assembly}{ensamblador}, NASM soporta etiquetas
locales, evaluación de \gls{expression} aritméticas (con algunos operadores más
que \gls{GAS}), cadenas en UTF-8 \parencite{UTF-8}, \gls{compilation}
condicional, definición de símbolos con valores arbitrarios, y generación de
código con bucles. Cabe destacar su potente sistema de \glspl{macro}, que
permite \glspl{macro} variadicas, sobrecarga (definición de múltiples
\glspl{macro} con el mismo nombre) según el número de argumentos, etiquetas
locales a la \gls{macro}, y múltiples funcionalidades para procesar los
argumentos.

NASM permite optimizar el código generado realizando múltiples pasadas sobre
este. Sin embargo, para realizar esto, necesita conocer el tamaño de todos los
datos e \glspl{instruction} durante la primera pasada. Debido a esto, no permite
el uso de \glspl{forward-reference} en situaciones en las que el valor de estas
puede afectar al tamaño de los elementos generados. \parencite{NASM-manual}

% TODO: comparativa ensambladores

\section{Técnicas de análisis sintáctico}\label{sec:parser-techniques}

Un \gls{compiler} se compone principalmente de dos partes: análisis sintáctico y
análisis semántico. El análisis sintáctico (\textit{parsing}) se encarga de
transformar el código a una representación intermedia, mientras que el análisis
semántico se encarga de, utilizando la representación intermedia, verificar que
el \gls{program} cumple la semántica del \glsdisp{programming language}{lenguaje}
y generar el código compilado. \parencite{dragon-book}

Al tratar con la semántica del \glsdisp{programming language}{lenguaje}, no
existen técnicas generales para realizar el análisis semántico. Sin embargo,
para realizar el análisis sintáctico, existen múltiples técnicas basadas en el
uso de \glspl{grammar} libres de contexto. Una \gls{grammar} es una descripción
formal de la sintaxis de un lenguaje, y está formada por:
\parencite{dragon-book}

\begin{itemize}
    \item Un conjunto de símbolos terminales (también conocidos como
    \glspl{token}), que representan los símbolos que pueden aparecer en las
    cadenas del lenguaje.
    \item Un conjunto de símbolos no terminales, que representan estructuras
    sintácticas del lenguaje (conjuntos de cadenas).
    \item Un símbolo no terminal denominado axioma, que representa todas las
    cadenas del lenguaje.
    \item Un conjunto de producciones, que representan las reglas según las
    cuales un símbolo no terminal se puede reemplazar por una secuencia de
    símbolos terminales y/o no terminales.
\end{itemize}

Una gramática genera cadenas de caracteres aplicando las producciones a una
secuencia de símbolos, empezando por el axioma y continuando hasta que la
secuencia de símbolos solo contenga símbolos terminales. Este proceso se puede
representar con un árbol síntáctico, en el cual los nodos representan símbolos
(siendo la raíz el axioma, los \glspl{internal node} símbolos no terminales, y
las \glsdisp{leaf node}{hojas} símbolos terminales), y las relaciones entre un
nodo y sus hijos representan la aplicación de una producción determinada. La
tarea de un analizador sintáctico (\gls{parser}) consiste en determinar el árbol
sintáctico que produce la cadena de entrada. \parencite{dragon-book}

Las técnicas para realizar el análisis sintáctico se pueden agrupar en dos
grandes familias: análisis descendence (\textit{top-down}) y análisis ascendente
(\textit{bottom-up}). La diferencia entre estas familias está en cómo generan el
árbol sintáctico. Un analizador descendente empieza por la raíz (axioma) y
determina las producciones que se tienen que aplicar para obtener la cadena de
entrada, realizando una búsqueda en profundidad. Un analizador ascendente, en
cambio, empieza con los nodos hoja del árbol y determina cómo agruparlos en
símbolos no terminales siguiendo las producciones hasta llegar a la raíz.
\parencite{dragon-book}

\subsection{Análisis descendente}

Dentro de los analizadores descendentes, la técnica más general y utilizada es
el análisis descendente recursivo, que consiste en un \gls{program} que modela la
\gls{grammar} mediante un conjunto de funciones mutuamente recursivas, cada una
de las cuales representa un símbolo no terminal. Analizando la \gls{grammar}, se
puede crear un analizador predictivo en el cual el siguiente símbolo de la
entrada a procesar permite determinar sin ambigüedades el flujo de control en
cada función. \parencite{dragon-book}

Gracias a su sencillez, esta técnica típicamente se implementa a mano, aunque
también existen herramientas como ANTLR \parencite{ANTLR} que permiten generar
un analizador automáticamente a partir de una \gls{grammar} (conocido como
generador de \glspl{parser} \parencite{dragon-book}).

También existen otros métodos para crear analizadores descendentes
recursivos a partir de una \gls{grammar}, como los combinadores de
\glspl{parser}. Este método consiste en definir un analizador sintáctico como
combinación de otros \glspl{parser} más simples, utilizando diversos operadores
que permiten realizar combinaciones muy expresivas
\parencite{parser-combinators}. En la actualidad, existen muchas bibliotecas que
cuentan con estos operadores ya implementados, como Parsec \parencite{parsec}, Chumsky
\parencite{chumsky}, y pom \parencite{pom}.

La principal ventaja de crear un analizador descendente recursivo manualmente es
su gran flexibilidad, ya que ofrece control total sobre el funcionamiento del
analizador sintáctico. Esto permite utilizar información obtenida durante la
ejecución del \gls{compiler} para decidir cómo procesar la entrada, permitiendo
un cierto grado de sensibilidad al contexto. Además, esto ayuda a generar buenos
mensajes de error y a aplicar estrategias de recuperación de errores
\parencite{errors-clang}. El funcionamiento de esta técnica también es más
similar a la forma en la que se escribe el código, lo que facilita obtener
información de bloques grandes de código aún en presencia de errores sintácticos
\parencite{resilient-LL-parsing}. Muchos de los \glspl{programming language}
utilizados en la actualidad, como C y C++, utilizan este método
\parencite{parser-types-survey}.

El principal problema de crear un analizador descendente recursivo a mano es su
complejidad, requiriendo mucho tiempo para su desarrollo, además de que las
modificaciones de la \gls{grammar} también son complicadas. Los combinadores de
\glspl{parser}, en cambio, facilitan mucho estos procesos, aunque sacrifican un
poco de rendimiento para lograrlo. Esto hace que los combinadores de
\glspl{parser} sean un método ideal para realizar prototipos, durante los cuales
la \gls{grammar} puede variar, mientas que crear un analizador manualmente es
una mejor opción para un producto final cuando el rendimiento se vuelve un
problema.

Estas técnicas, en general, son muy efectivas. Sin embargo, para \glspl{grammar}
que cumplen ciertas propiedades (\glspl{operator-precedende grammar}) típicas de
\gls{expression} aritméticas, especialmente cuando cuentan con muchos niveles de
precedencia, existen técnicas más eficientes. Estas técnicas se pueden agrupar
en los \gls{parser} de precedencia de operadores
\parencite{operator-precedence-parser}. Un ejemplo de estas técnicas es el
análisis Pratt \parencite{pratt-parsing-paper}
\parencite{pratt-parsing-example}, que representa los niveles de precedencia con
funciones para comparar operadores. Gracias a esto, se puede evitar representar los niveles de
precedencia en la propia \gls{grammar}, permitiendo utilizar una representación
más natural de la \gls{grammar}.

\subsection{Análisis ascendente}

Dentro de los analizadores ascendentes, existen muchas técnicas distintas. La
mayoría de estas técnicas pertenecen a la familia LR, como SLR, LALR, o GLR.
Todas las técnicas en esta subfamilia utilizan un algoritmo similar: cuentan con
una pila inicialmente vacía y un autómata finito (representado por una tabla)
que, según su estado, determina su siguiente estado y la operación a aplicar a
la entrada y pila. Esta operación puede ser añadir un símbolo de la entrada a la
cima de la pila o agrupar los últimos símbolos de la pila en un símbolo no
terminal según una producción determinada. La diferencia entre estos métodos
está en cómo se construye el autómata finito y el conjunto de \glspl{grammar}
que pueden representar, siendo SLR un método simple que puede representar pocas
\gls{grammar}, mientras que LALR puede representar más \glspl{grammar}, pero es
más complejo de construir. \parencite{dragon-book} GLR es una generalización de
las técnicas LR para permitir procesar \glspl{ambiguous grammar}
\parencite{GLR-algorithm}.

Debido a la gran cantidad de estados que puede tener el autómata, estos
analizadores típicamente se contruyen automáticamente a partir de la
\gls{grammar} con un generador de \glspl{parser} \parencite{dragon-book}.
Existen muchas herramientas de este tipo como Lex y yacc \parencite{yacc}
(basado en LALR), sus versiones modernas de GNU Flex y bison \parencite{bison}
(basado en LALR por defecto, aunque pueden utilizar GLR), o Tree-sitter
\parencite{tree-sitter} (basado en GLR).

\section{Mensajes de error}\label{sec:error-messages}

En la actualidad, los mensajes de error de los \gls{compiler} tienen muchas
formas y calidades. Un \gls{compiler} con buenos mensajes de error puede ayudar
mucho a solucionar los problemas más rápidamente, ya que permite obtener toda la
información necesaria para solucionar el problema del propio mensaje de error.

Un ejemplo de malos mensajes de error es el \gls{compiler} de C\#, cuyos mensajes
de error se pueden ver en la \figureref{error_csharp}. Como se puede ver, estos
mensajes contienen muy poca información sobre el problema. Esto dificulta
corregir los errores, especialmente para alguien que está intentando aprender el
\glsdisp{programming language}{lenguaje}.

\svgfigure[0.9]{error_csharp}{Mensaje de error del compilador de C\#}

Actualmente, el compilador de Rust \parencite{Rust} es uno de los
compiler que mejores mensajes de error genera. La \figureref{error_rust}
muestra un ejemplo de estos mensajes, obtenido en un caso de uso real. Como se
puede observar, este mensaje cuenta con varios componentes importantes:

\begin{itemize}
    \item Un mensaje de error con una explicación clara del error.
    \item La localización en el código que ha producido el fallo.
    \item Información adicional sobre el error y su causa, resaltada sobre el
    propio código.
    \item Un mensaje de ayuda que indica una posible solución del error,
    incluyendo una explicación y las modificaciones necesarias resaltadas sobre
    el código.
    \item Una forma de acceder a una explicación detallada sobre el problema.
\end{itemize}

Además de esto, la información está resaltada con diversos colores e indicadores
y se deja cierto espacio en blanco para separar las diferentes partes, lo cual
ayuda a leer rápidamente la información más importante para solucionar el
problema y filtrar la que no se necesita.

\svgfigure[0.9]{error_rust}{Mensaje de error del compilador de Rust}

\FloatBarrier

\section{Lenguajes de programación}\label{sec:programming-languages}

Actualmente, las herramientas utilizadas durante el desarrollo de
\textit{software} para los diversos \glsdisp{programming language}{lenguajes},
como \glspl{compiler}, \glspl{formatter}, o \glspl{linter}, se implementan en
muchos \glspl{programming language} distintos.

Para los \glsdisp{programming language}{lenguajes} de bajo nivel, como C/C++ o
\glsdisp{assembly}{ensamblador}, típicamente las herramientas se implementan en
\glsdisp{programming language}{lenguajes} de bajo nivel, comúnmente el mismo
para el que se utilizan. Esto es debido a su mayor rendimiento, ya que el
\textit{software} realizado en estos \glsdisp{programming language}{lenguajes}
sule tener grandes bases de código. La mayoría de herramientas de estos
\glsdisp{programming language}{lenguajes}, como \progref{gcc} y la
\textit{suite} de \verb!clang! (\verb!clang!, \verb!clang-tidy! y
\verb!clang-format!) \parencite{clang}, utilizan C/C++. Una excepción importante
a esto es \progref{conan}, el gestor de paquetes de C/C++, que está escrito en
un \glsdisp{programming language}{lenguaje} de alto nivel (Python).

En los \glsdisp{programming language}{lenguajes} de alto nivel, como Python o
\gls{TS}, las herramientas tradicionalmente se han implementado en
\glsdisp{programming language}{lenguajes} de alto nivel para facilitar su
desarrollo. Este es el caso de \progref{pylint}, \progref{black} o el
\gls{compiler} de \gls{TS} \parencite{tsc}. Sin embargo, recientemente muchas de
estas herramientas se están reescribiendo en \glsdisp{programming
language}{lenguajes} de más bajo nivel como Rust o Go debido a la mejora
significativa de rendimiento que ofrecen estos \glsdisp{programming
language}{lenguajes} \parencite{typescript-go}. Algunos ejemplos de herramientas
que han tenido éxito con esto son \progref{ruff} y \progref{uv}.
