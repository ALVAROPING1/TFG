\chapter{Diseño}\label{chap:design}

Este capítulo describirá el diseño del sistema y está dividido en dos
secciones. La primera (\sectionref{solution_study}), hará un estudio de las
alternativas posibles para el desarrollo, llegando a una solución que cumple con
los requisitos establecidos en la \sectionref{requirements}. La segunda
(\sectionref{architecture}), describirá la arquitectura del compilador,
incluyendo sus componentes y decisiones de diseño.

\section{Estudio de la solución final}\label{sec:solution_study}

Para realizar el estudio de la solución, se ha tenido en cuenta la elicitación
de requisitos (\sectionref{requirements}), ya que esto determina las capacidades
y restricciones del sistema y, por lo tanto, el diseño dependerá de ellos.

\subsection{Lenguaje de programación}\label{subsec:language}

Según los requisitos \sreqref[]{NF-chrome} a \sreqref[]{NF-JS}, el sistema
debe poder ejecutarse en navegador y Node.js, y tener una \gls{API} para
\gls{JS}. Esto deja dos alternativas principales: utilizar \gls{JS} o
\gls{wasm}.

\gls{JS} es el lenguaje estándar utilizado por los navegadores y Node.js, por lo
que es la primera opción para el desarrollo del sistema. El principal problema
de esta opción es el rendimiento: \gls{JS} es un lenguaje interpretado, lo cual
lo hace significativamente más lento que otros lenguajes compilados. Esto es
debido a que estos lenguajes pueden generar código con mejor rendimiento al
poder ejecutarse directamente en el \gls{processor} (ejecución nativa), sin el
sobrecoste de una máquina virtual o intérprete. El \sreqref{NF-velocidad}
indica que el sistema debe ser rápido, por lo que sería preferible utilizar un
lenguaje que soporte la ejecución nativa.

Otro problema de \gls{JS} es su \gls{type system}. \gls{JS} utiliza un sistema de
tipos \glsdisponly{dynamic typing}{dinámico} y \glsdisponly{weak typing}{débil},
con muchas conversiones implícitas de tipos \parencite{js-type-system}. Esto
dificulta razonar sobre el código, ya que no garantiza que los tipos de datos
serán los esperados, y permite que aparezcan errores muy difíciles de encontrar.
Además, las conversiones implícitas de tipos permiten que los errores se
propaguen antes de ser detectados, dificultando encontrar su origen. Debido a
esto, sería preferible utilizar un lenguaje con un \gls{type system}
\glsdisponly{static typing}{estático} y \glsdisponly{strong typing}{fuerte}.

\gls{wasm} \parencite{wasm} es un nuevo estándar soportado por los navegadores y
Node.js que permite la ejecución de \glspl{program} compilados de forma casi
nativa, obteniendo un rendimiento significativamente mejor que \gls{JS}. Muchos
lenguajes como C \parencite{C}, C++ \parencite{cpp}, o Rust \parencite{Rust}
soportan la compilación a \gls{wasm}, posibilitando su uso para desarrollar el
sistema. Todos estos lenguajes tienen un rendimiento prácticamente idéntico, por
lo que se necesitan analizar otras características para elegir la mejor opción.

En la actualidad, la mayoría de las vulnerabilidades de seguridad en software
están causadas por problemas en el uso y gestión de \gls{memory}
\parencite{microsoft-memory-safety} \parencite{mozilla-memory-safety}. Utilizar
un lenguaje que ofrezca \gls{memory safety} permite evitar estos fallos, y
actualmente se está instando por la adopción de este tipo de lenguajes
\parencite{CISA-memory-safety} \parencite{whitehouse-memory-safety}. De los
lenguajes anteriormente mencionados, el único que cumple con esta característica
es Rust \parencite{C-memory-safety}. % TODO: añadir cita sobre rust siendo memory safe

Otra característica importante es la capacidad de generación de \glspl{binding}.
Estos son necesarios para permitir que el código \gls{JS} pueda utilizar un
módulo \gls{wasm} como una \gls{library}, ofreciendo una \gls{API} para \gls{JS}
como pide el \sreqref{NF-JS}. Existen herramientas como Emscripten (C/C++)
\parencite{Emscripten} o Wasm-pack (Rust) \parencite{Wasm-pack} que realizan
esto para todos los lenguajes mencionados anteriormente. Cabe destacar que
Emscripten es algo más manual, ya que, por ejemplo, requiere redefinir los tipos
de los argumentos y valor de retorno para cada función que se quiera ejecutar
desde \gls{JS}. Wasm-pack, en cambio, es capaz de generar automáticamente todo
el código necesario para poder utilizar las funciones definidas en el módulo
\gls{wasm} como si se tratasen de funciones nativas de \gls{JS}.

La \tableref{language-comparison} resume las características analizadas. Teniendo
en cuenta todas estas características, se ha elegido Rust como lenguaje a
utilizar para el desarrollo del sistema.

\begin{table}[htb]
    \caption{Comparación de los lenguajes de programación}
    \label{tab:language-comparison}
    \begin{adjustbox}{max width=\textwidth}
        \begin{tabular}{@{}lcccc@{}}
            \toprule
            Lenguaje            & \gls{JS}   & C          & C++        & Rust \\
            \midrule
            Ejecución nativa    &            & \checkmark & \checkmark & \checkmark \\
            \Gls{memory safety} & \checkmark &            &            & \checkmark \\
            Buena interoperabilidad con \gls{JS}
                                & \checkmark &            &            & \checkmark \\
            \Gls{static typing} &            & \checkmark & \checkmark & \checkmark \\
            \Gls{strong typing} &            & \checkmark & \checkmark & \checkmark \\
            \bottomrule
        \end{tabular}
    \end{adjustbox}
\end{table}

\FloatBarrier

\subsection{Análisis sintáctico}\label{subsec:parser}

Un compilador se compone principalmente de dos partes: análisis sintáctico y
análisis semántico. El análisis sintáctico se encarga de transformar el código a
un \gls{AST}, mientras que el análisis semántico (\textit{parsing}) se encarga
de, utilizando el \gls{AST}, verificar que el \gls{program} cumple la semántica
del lenguaje y generar el código compilado. \parencite{dragon-book}

Para realizar el análisis sintáctico, actualmente existen múltiples métodos:
utilizar un generador de \glspl{parser}, utilizar un combinador de
\glspl{parser}, o crear un analizador sintáctico manualmente.
\parencite{dragon-book} \parencite{parser-combinators}

Un generador de \glspl{parser} permite generar automáticamente un analizador
sintáctico a partir de la definición de una \gls{grammar}, facilitando mucho su
desarrollo \parencite{dragon-book}. Existen muchas herramientas de este tipo
como Lex y yacc \parencite{yacc}, sus versiones modernas de GNU Flex y bison
\parencite{bison}, ANTLR \parencite{ANTLR}, o Tree-sitter
\parencite{tree-sitter}. Durante la generación del analizador sintáctico, estas
herramientas realizan un procesado de la gramática que verifica ciertas
restricciones y les permite generar un analizador sintáctico eficiente.
Generalmente, esto es útil, ya que ayuda a detectar errores en la \gls{grammar}.
Sin embargo, el \sreqref{NF-def-instrucciones} indica que la definición de la
sintaxis de las instrucciones debe estar contenida en la \gls{ISA}, y, por lo
tanto, la \gls{grammar} es parcialmente desconocida en \gls{compile-time}. Esto
imposibilita el uso de estas herramientas.

El siguiente método es crear el analizador sintáctico manualmente, típicamente
utilizando la técnica de descenso recursivo \parencite{dragon-book}. Este es el
método más flexible, ya que da control total sobre el funcionamiento del
analizador sintáctico. Esto permite utilizar información obtenida durante la
ejecución del \gls{compiler} para decidir cómo procesar la entrada. Además, esto
ayuda a generar buenos mensajes de error, como pide el
\sreqref{FN-errores-base}, y a aplicar estrategias de recuperación de errores
\parencite{errors-clang}. Muchos de los \glspl{programming language} utilizados
en la actualidad, como C y C++, utilizan este método
\parencite{parser-types-survey}. El principal problema de este método es su
complejidad, requiriendo mucho tiempo para su desarrollo.

El último método es utilizar un combinador de \glspl{parser}. Este método
permite definir un \gls{parser} como combinación de otros \glspl{parser} más
simples \parencite{parser-combinators}. Esta técnica permite crear un analizador
sintáctico de una forma similar a como se haría un \gls{parser} descendente
recursivo, obteniendo así muchos de los beneficios de la creación manual, pero
simplificando mucho el proceso gracias a un mayor nivel de abstracción. Algunas
de las herramientas de este tipo son Parsec \parencite{parsec}, Chumsky
\parencite{chumsky}, y pom \parencite{pom}.

Teniendo en cuenta esto, se ha elegido Chumsky como biblioteca para la creación
del analizador sintáctico debido a varios motivos.

\begin{itemize}
    \item Es una biblioteca de combinadores de \glspl{parser}.
    \item Está diseñada para generar buenos mensajes de error, y ser lo más
    rápida posible sin sacrificar la calidad de estos mensajes.
    \item Tiene soporte para Rust, el \gls{programming language} elegido
    anteriormente.
\end{itemize}

\subsection{Análisis semántico}\label{subsec:compiler}

% TODO: AST->representación intermedia
Como se ha explicado en la sección anterior, el análisis semántico
(\textit{parsing}) se encarga de, utilizando el \gls{AST}, verificar que el
\gls{program} cumple la semántica del lenguaje y generar el código compilado.

Aunque lógicamente el analizador semántico toma como entrada el resultado del
analizador sintáctico, ambos componentes se pueden mezclar para evitar
representar el \gls{AST} explícitamente, realizando ambas tareas simultáneamente
y aumentando la eficiencia del \gls{compiler}. Sin embargo, esto dificulta la
tarea del analizador semántico al no poder explorar libremente el \gls{AST}. En
la práctica, esto solo es posible para lenguajes y \glspl{compiler} muy simples.
Debido a esto, se ha optado por construir un \gls{AST} explícitamente para tener
una mayor flexibilidad. \parencite{compiler-design}

Otra decisión importante para el desarrollo del analizador sintáctico es el
tratamiento de las \glspl{forward-reference}, pedidas por el
\sreqref{FN-referencias}. Si el compilador escanease el código secuencialmente,
al llegar a una etiqueta usada antes de su definición no podría realizar la
traducción al desconocer el valor de esta etiqueta. Este problema se puede
solucionar permitiendo la modificación de código ya traducido o realizando
múltiples pasadas sobre el código. En el primer caso, cuando se encuentra una
etiqueta sin definir, se registra en una tabla, y cuando se encuentra su
definición se utiliza esta tabla para corregir sus usos. En el segundo caso, se
realizan al menos dos pasadas por el código: en la primera se determina el
espacio que ocupará cada elemento para asignar sus direcciones de \gls{memory} y
se obtienen las definiciones de las etiquetas, mientras en la segunda pasada se
realiza la traducción del código. \parencite{assembler-design}

Realizar una única pasada es más eficiente, pero también es menos flexible y
añade más complejidad al código. Actualmente, muchos \glspl{assembler} optan por
realizar múltiples pasadas, ya que da más flexibilidad para implementar
funcionalidades avanzadas que no pueden ser implementadas en un \gls{assembler}
de una única pasada \parencite{assembler-design}. Teniendo en cuenta esto, se ha
elegido crear un compilador de múltiples pasadas para permitir una mayor
flexibilidad para añadir nuevas funcionalidades en el futuro.

\section{Arquitectura del compilador}\label{sec:architecture}

\noindent
El compilador a desarrollar está formado por tres componentes principales:

\begin{enumerate}
    \item Gestor de la arquitectura: encargado de validar y cargar la \gls{ISA}
    a utilizar durante la compilación.
    \item Analizador sintáctico: encargado de realizar el análisis sintáctico del
    código ensamblador para permitir su análisis y traducción.
    \item Analizador semántico: encargado de realizar el análisis semántico del
    código ensamblador, y traducirlo a un formato que pueda ser utilizado por
    CREATOR.
\end{enumerate}

La \figureref{components} muestra el modelo de componentes UML
\parencite{UMLSpec} del compilador, incluyendo las relaciones entre los
componentes.

\graphicfigure[0.95]{components}{Modelo de componentes del compilador}

\printcomptemplate

\begin{component}{Gestor de la arquitectura}
    {Valida y carga la \gls{ISA}, permitiendo su uso en los otros componentes} % Rol
    {} % Dependencias
    {Especificación de la \gls{ISA}} % Datos de entrada
    {\gls{ISA} validada} % Datos de salida
    {FN-ISA} % Origen
    El gestor de la arquitectura se encarga de validar y cargar la
    especificación de la \gls{ISA} proporcionada, generando diferentes errores
    en caso de fallo en la validación para evitar el uso de una \gls{ISA} mal
    definida.
\end{component}

\begin{component}{Lexer}
    {Transforma el código del \gls{program} en una secuencia de \glspl{token}} % Rol
    {} % Dependencias
    {\Gls{program} a analizar} % Datos de entrada
    {Secuencia de \glspl{token}} % Datos de salida
    {FN-compilar-instrucciones, FN-compilar-pseudo, FN-directivas,
    FN-directivas-datos, FN-escape, FN-floats, FN-comentarios,
    FN-constantes, FN-expr-operadores, FN-errores-base} % Origen
    El lexer recibe el código del \gls{program} a
    \glsdisp{compilation}{compilar} y realiza un análisis léxico, transformado
    la secuencia de caracteres que representa el \gls{program} en una secuencia
    de \glspl{token} que puede ser utilizada por los demás componentes del
    analizador sintáctico.
\end{component}

\begin{component}{Parser de expresiones}
    {Realiza el análisis sintáctico de una \gls{expression}} % Rol
    {} % Dependencias
    {Secuencia de \glspl{token}} % Datos de entrada
    {\gls{AST} correspondiente a una \gls{expression}} % Datos de salida
    {FN-floats, FN-constantes, FN-expr-operadores,
    FN-expr-instruccion, FN-expr-directiva, FN-errores-base} % Origen
    El \gls{parser} de expresiones realiza el análisis sintáctico de una
    \gls{expression}, transformando la secuencia de \glspl{token} que la
    representa en un \gls{AST} para su posterior evaluación. Este componente
    tiene en cuenta el nivel de precedencia de los operadores al generar el
    \gls{AST}.
\end{component}

\begin{component}{Parser}
    {Realiza el análisis sintáctico de un \gls{program}} % Rol
    {} % Dependencias
    {Secuencia de \glspl{token}} % Datos de entrada
    {\gls{AST} correspondiente al \gls{program}} % Datos de salida
    {FN-compilar-instrucciones, FN-compilar-pseudo, FN-directivas,
    FN-directivas-datos, FN-num-etiquetas, FN-errores-base} % Origen
    El \gls{parser} realiza el análisis sintáctico de un \gls{program},
    transformando la secuencia de \glspl{token} que lo representa en un
    \gls{AST} para su posterior análisis semántico.
\end{component}

\begin{component}{Parser de instrucciones}
    {Realiza el análisis sintáctico de una \gls{instruction}} % Rol
    {} % Dependencias
    {Secuencia de \glspl{token} y definición de la sintaxis de la \gls{instruction}} % Datos de entrada
    {\gls{AST} correspondiente a la \gls{instruction}} % Datos de salida
    {FN-compilar-instrucciones, FN-compilar-pseudo,
    FN-mult-defs-instruccion, FN-errores-base} % Origen
    El \gls{parser} de \gls{instruction} realiza el análisis sintáctico de los
    argumentos de una \gls{instruction} según una cadena que define su sintaxis.
\end{component}

\begin{component}{Evaluador de pseudo-instrucciones}
    {Transforma una \gls{pseudo-instruction} en una secuencia de \glspl{instruction}} % Rol
    {} % Dependencias
    {\gls{AST} de la \gls{pseudo-instruction}, \gls{ISA}, \gls{symbol table}, y dirección de la siguiente \gls{instruction}} % Datos de entrada
    {Secuencia de \glspl{AST} correspondientes a las \glspl{instruction} por las que se tiene que remplazar la \gls{pseudo-instruction}} % Datos de salida
    {FN-compilar-pseudo, FN-errores-base} % Origen
    El evaluador de \glspl{pseudo-instruction} se encarga de transformar las
    \glspl{pseudo-instruction} en una secuencia de \glspl{instruction} con el
    mismo comportamiento.
\end{component}

\begin{component}{Compilador}
    {Realiza el análisis semántico y traducción de un \gls{program} \glsdisp{assembly}{ensamblador}} % Rol
    {} % Dependencias
    {\gls{ISA} y código del \gls{program} a \glsdisp{compilation}{compilar}} % Datos de entrada
    {\Gls{program} \glsdisp{compilation}{compilado}} % Datos de salida
    {FN-compilar-instrucciones, FN-compilar-pseudo, FN-directivas,
    FN-secciones, FN-directivas-datos, FN-strings, FN-tamaño-ints,
    FN-floats, FN-alineamiento, FN-bibliotecas, FN-num-etiquetas,
    FN-referencias, FN-constantes, FN-expr-operadores,
    FN-mult-defs-instruccion, FN-errores-base, FN-bigints} % Origen
    El \gls{compiler} se encarga de realizar el análisis semántico y traducción
    de \glspl{program} \glsdisp{assembly}{ensamblador} según una \gls{ISA} dada.
    Durante este proceso, verifica que no haya ningún error en el código, y lo
    traduce a una representación simplificada que puede ser ejecutada en
    CREATOR.
\end{component}

\FloatBarrier

\subsubsection{Trazabilidad}

La matriz de trazabilidad permite verificar que todos los requisitos funcionales
están cubiertos por al menos un componente.

\begin{landscape}
    \traceabilityTable[p]{trazability-comp-f}{^[^R]}{^RS\37-FN}
        {Trazabilidad entre los requisitos funcionales y los componentes}
\end{landscape}

\FloatBarrier

\subsection{Analizador sintáctico}

El analizador sintáctico se compondrá principalmente de un \gls{lexer} y un
\gls{parser}. El \gls{lexer} se encargará de realizar un procesado inicial del
código para transformarlo en una secuencia de \glspl{token} (eliminando espacios
y comentarios), mientras que el \gls{parser} tomará esta secuencia de
\glspl{token} y la transformará en un \gls{AST} según la \gls{grammar} del
lenguaje. Además, también utilizará un \gls{parser} de \glspl{expression} para
realizar el análisis de las \glspl{expression} numéricas, el cual tendrá en
cuenta el nivel de prioridad de los diferentes operadores.

% TODO: errores sintácticos->semánticos porque son mejores para un usuario?

El \sreqref{FN-mult-defs-instruccion} indica que una instrucción puede tener
múltiples sintaxis distintas. Para permitir esto, se ha decidido que en el
análisis sintáctico inicial del código se interpretará la secuencia de
\glspl{token} entre el nombre de la instrucción y el siguiente salto de línea
como los argumentos de la instrucción en todos los casos, ya que es el
funcionamiento de GNU as. % TODO: cita
El último componente del analizador sintáctico, el \gls{parser} de
instrucciones, intentará realizar el análisis sintáctico de esta secuencia de
\glspl{token} según una especificación de la sintaxis de la instrucción,
devolviendo un error en caso de fallo.

A continuación se incluye la gramática en forma normal de Backus-Naur % TODO: cita
que utilizará el \gls{parser}. En general, se ha preferido simplificar la
gramática y tratar los posibles errores en el analizador semántico, ya que esto
permite generar errores con mayor información que pueden ser más útiles para un
usuario, como pide el \sreqref{FN-errores-base}. % TODO: mencionar simplificación saltos de línea?

\newcommand{\EOL}{\textbackslash{}n}
\newcommand{\verbEOL}{`\texttt{\textbackslash{}n}'}

\begin{tcolorbox}[enhanced jigsaw,breakable,top=4mm]
    \begin{grammar}
       <label> ::= <identifier> `:'

       <directive\_name> ::= `.' <identifier>

       <labels> ::= $\lambda$ | <label> <labels>

       <statement> ::= <labels> <directive> `\EOL'
                  \alt <labels> <instruction> `\EOL'

       <directive> ::= <directive\_name> <args>
       
       <args> ::= $\lambda$ | <arg> <args\_list>

       <args\_list> ::= $\lambda$ | `,' <arg> <args\_list>

       <arg> ::= <expression> | <string>

       <instruction> ::= <identifier> <tokens>

       <tokens> ::= $\lambda$ | <not\_newline> <tokens>

       <not\_newline> ::= \textit{Cualquier token excepto \verbEOL}
    \end{grammar}
\end{tcolorbox}

\subsection{Analizador semántico}
