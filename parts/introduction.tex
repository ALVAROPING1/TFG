\chapter{Introducción}\label{chap:introduction}
\section{Motivación}\label{sec:motivation}



\section{Objetivos}\label{sec:objectives}

El principal objetivo de este proyecto es desarrollar un \gls{compiler} de
\gls{assembly} (\gls{assembler}) genérico que sustituya al actualmente utilizado
por el simulador CREATOR, añadiendo nuevas funcionalidades y mejorando sus
mensajes de error. Esto ayudará a los alumnos a aprender el \gls{assembly}.

\noindent
A partir de este objetivo principal, se pueden definir objetivos secundarios:

\begin{itemize}
    \item \textbf{O1:} Utilizar diferentes juegos de \glspl{instruction}.
    \item \textbf{O2:} \glsdisp{compilation}{Compilar} cualquier programa
    \glsdisp{assembly}{ensamblador} escrito con las \glspl{instruction}
    definidas.
    \item \textbf{O3:} Generar mensajes de error con mucha información que
    ayuden a los estudiantes comprender la causa del problema.
    \item \textbf{O4:} Ser compatible con las definiciones de
    \glsdisp{ISA}{arquitecturas} actualmente desarrolladas para CREATOR
    \item \textbf{O5:} Ser lo suficientemente flexible como para añadir nuevas
    funcionalidades, que permitan el uso de CREATOR en nuevas situaciones.
\end{itemize}

\section{Estructura del documento}\label{sec:structure}

Este documento estará formado por los siguientes capítulos:

\begin{itemize}
    \item \chapterref{introduction}, realiza una introducción al proyecto,
    explicando sus motivaciones y objetivos. Además, también contiene una
    descripción de los contenidos del documento.
    \item \chapterref{state-of-the-art}, analiza el estado de los simuladores y
    \glspl{assembler} existentes, las diferentes técnicas utilizadas para
    desarrollar analizadores sintácticos, los mensajes de error generados por
    diversos \glspl{compiler}, y los \glspl{programming language} utilizados
    para el desarrollo de estas herramientas.
    \item \chapterref{analysis}, presenta el proyecto y define sus requisitos y
    casos de uso.
    \item \chapterref{design}, explica el diseño y arquitectura del sistema
    creado, y sus componentes.
    \item \chapterref{implementation}, presenta los detalles de implementación
    del sistema, y explica cómo realizar su despliegue.
    \item \chapterref{validation}, indica cómo se ha verificado el correcto
    desarrollo del sistema desarrollando, detallando las pruebas y comparándolo
    con otras herramientas.
    \item \chapterref{planning}, describe la planificación seguida en el
    proyecto y su presupuesto. También discute el marco regulador y entorno
    socioeconómico en el que se desarrolla el mismo.
    \item \chapterref{conclusions}, expone las conclusiones obtenidas con la
    realización del proyecto, contribuciones adicionales realizadas durante el
    mismo, y los trabajos futuros que podrían mejorar el sistema.
\end{itemize}
