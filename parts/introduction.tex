\chapter{Introducción}\label{chap:introduction}

En este primer capítulo se va a presentar el proyecto. Primero se explicará su
motivación (\sectionref{motivation}). Tras esto, los objetivos a
lograr (\sectionref{objectives}). Finalmente, se indicará la estructura de este
documento (\sectionref{structure}).

\section{Motivación}\label{sec:motivation}

Este Trabajo Fin de Grado (TFG) se ha desarrollado como parte de una beca de
colaboración con el Departamento de Informática financiada por el Ministerio de
Educación, Formación Profesional y Deportes. Dentro de esta colaboración, el TFG
se ha enmarcado dentro del contexto del proyecto investigación ``Entorno de
desarrollo integrado para formación e investigación en procesadores RISC-V''
financiado por la Agencia Estatal de Investigación dentro de la convocatoria de
proyectos ``Prueba de Concepto'' 2023-PERTE CHIP, dentro del cual se ha
desarrollado el \gls{compiler} de \glsdisp{assembly}{ensamblador} desarrollado
en este TFG.

La programación es el acto de componer secuencias de instrucciones que indiquen
a un \gls{computer} cómo realizar una tarea. Esto se realiza describiendo esta
secuencia en un \gls{programming language}, creando un \gls{program}. Según el
nivel de abstracción del \gls{computer}, los \glspl{programming language} se
pueden clasificar en un espectro, donde los \glspl{low-level language} utilizan
pocas o ninguna abstracción mientras que los \glspl{high-level language}
utilizan muchas abstracciones.

Aunque estas abstracciones ayudan al desarrollo de \glspl{program}, en todos los
casos estos acaban siendo ejecutados en un \gls{computer}. Es por esto que el
funcionamiento del mismo siempre tiene una influencia en todos los
\glsdisp{programming language}{language}, y determina las consequencias que
tiene el uso de las diferentes abstracciones ofrecidas por los \glspl{high-level
language}. Debido a esto, para diseñar \glspl{program} que sean capaces de
utilizar eficientemente todos los recursos de un \gls{computer}, es fundamental
conocer su funcionamiento.

Un \gls{computer} es una combinación de \textit{hardware} y \textit{software},
que forman la \glsdisp{ISA}{arquitectura} del mismo. El \textit{hardware}
determina las capacidades del \gls{computer}, mientras que el \textit{software}
indica cómo se utilizan estas capacidades para realizar una tarea. Conocer ambos
componentes es fundamental para entender el funcionamiento del \gls{computer},
qué es capaz de realizar, y cómo lo hace.

El \textit{software} que puede ejecutar directamente un \gls{computer} se
escribe en una familia de \glspl{low-level language} conocidos como
``\glspl{assembly}''. Para simplificar la electrónica y aumentar el rendimiento
del \gls{computer}, estos están formados por una secuencia de
\glspl{instruction} atómicas simples que operan en unos pocos datos. Para
ejecutar cualquier \gls{program} escrito en un \glsdisp{programming
language}{lenguaje} de mayor nivel es necesario convertirlos a uno de estos
\glspl{assembly} mediante un proceso conocido como ``\gls{compilation}''.

Debido a todo esto, conocer el funcionamiento de los \glspl{assembly} es
fundamental para cualquier ingeniero informático, ya que permite entender el
funcionamiento de un \gls{computer} y utilizarlo de forma efectiva para llevar a
cabo una tarea.

Actualmente, para enseñar estos \glspl{assembly} se utilizan simuladores, que
permiten al usuario visualizar el estado del \gls{computer} a medida que ejecuta
las \glspl{instruction} y pausar su ejecución en cualquier momento, de una forma
similar a las herramientas de depuración desarrolladas para los
\glspl{high-level language}. Un componente importante de estos simuladores es su
\gls{assembler} (también conocido como \gls{compiler}), encargado de convertir
la representación textual del código en \gls{assembly} a una representación que
pueda ser ejecutada en el simulador. Para permitir a los usuarios entender el
funcionamiento del \glsdisp{programming language}{lenguaje}, es necesario que
este \gls{assembler} detecte acciones no permitidas, y sea capaz de producir
mensajes de error que ayuden a los usuarios a entender y solucionar estos
problemas. Además, este \gls{assembler} tiene que ser lo suficientemente
flexible como para permitir realizar todas las acciones permitidas por un
\gls{computer} real.

Además, para que la enseñanza pueda seguir al rápido desarrollo de las
\glsdisp{ISA}{arquitecturas} de un \gls{computer}, estos simuladores se han
vuelto genéricos: permiten modificar muchos parámetros del \gls{computer}
simulado mediante un fichero de configuración, con pocos (o ningún) cambio en el
código. Esto permite simular múltiples \glsdisp{ISA}{arquitecturas} diferentes
en una misma herramienta sin necesidad de desarrollar un simulador completo para
cada una.

% TODO: Hablar de FOSS?

Por lo tanto, proponemos desarrollar un \gls{assembler} genérico para el simulador
CREATOR que sea robusto, lo suficientemente flexible como para permitir el uso
de funcionalidades avanzadas, y esté centrado en producir buenos mensajes de
error que ayuden a los usuarios a aprender el \gls{assembly}.

\section{Objetivos}\label{sec:objectives}

El principal objetivo de este proyecto es desarrollar un \gls{compiler} de
\gls{assembly} (\gls{assembler}) genérico que sustituya al actualmente utilizado
por el simulador CREATOR, añadiendo nuevas funcionalidades y mejorando sus
mensajes de error. Esto ayudará a los alumnos a aprender el \gls{assembly}.

\noindent
A partir de este objetivo principal, se pueden definir objetivos secundarios:

\begin{itemize}
    \item \textbf{O1:} Utilizar diferentes juegos de \glspl{instruction}.
    \item \textbf{O2:} \glsdisp{compilation}{Compilar} cualquier programa
    \glsdisp{assembly}{ensamblador} escrito con las \glspl{instruction}
    definidas.
    \item \textbf{O3:} Generar mensajes de error con mucha información que
    ayuden a los estudiantes comprender la causa del problema.
    \item \textbf{O4:} Ser compatible con las definiciones de
    \glsdisp{ISA}{arquitecturas} actualmente desarrolladas para CREATOR
    \item \textbf{O5:} Ser lo suficientemente flexible como para añadir nuevas
    funcionalidades, que permitan el uso de CREATOR en nuevas situaciones.
\end{itemize}

\section{Estructura del documento}\label{sec:structure}

Este documento estará formado por los siguientes capítulos:

\begin{itemize}
    \item \chapterref{introduction}, realiza una introducción al proyecto,
    explicando sus motivaciones y objetivos. Además, también contiene una
    descripción de los contenidos del documento.
    \item \chapterref{state-of-the-art}, analiza el estado de los simuladores y
    \glspl{assembler} existentes, las diferentes técnicas utilizadas para
    desarrollar analizadores sintácticos, los mensajes de error generados por
    diversos \glspl{compiler}, y los \glspl{programming language} utilizados
    para el desarrollo de estas herramientas.
    \item \chapterref{analysis}, presenta el proyecto y define sus requisitos y
    casos de uso.
    \item \chapterref{design}, explica el diseño y arquitectura del sistema
    creado, y sus componentes.
    \item \chapterref{implementation}, presenta los detalles de implementación
    del sistema, y explica cómo realizar su despliegue.
    \item \chapterref{validation}, indica cómo se ha verificado el correcto
    desarrollo del sistema desarrollando, detallando las pruebas y comparándolo
    con otras herramientas.
    \item \chapterref{planning}, describe la planificación seguida en el
    proyecto y su presupuesto. También discute el marco regulador y entorno
    socioeconómico en el que se desarrolla el mismo.
    \item \chapterref{conclusions}, expone las conclusiones obtenidas con la
    realización del proyecto, contribuciones adicionales realizadas durante el
    mismo, y los trabajos futuros que podrían mejorar el sistema.
\end{itemize}
