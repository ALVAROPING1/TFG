\chapter{Análisis}\label{chap:analysis}

Este capítulo describirá la solución propuesta y está dividido en tres
secciones. La primera (\sectionref{description}), hará una breve descripción del
proyecto. La segunda (\sectionref{requirements}), describirá los requisitos del
sistema a desarrollar. Por último, la tercera (\sectionref{usecases}),
especificará los diferentes casos de uso del sistema.

\section{Descripción del proyecto}\label{sec:description}

El objetivo de este proyecto es rehacer el \gls{compiler} utilizado por CREATOR,
para solucionar los problemas que presenta el compilador utilizado actualmente.
Este compilador será utilizado principalmente por estudiantes para programar en
\gls{assembly}. Debido a esto, es fundamental que el compilador genere buenos
mensajes de error que orienten a los estudiantes en su aprendizaje de este
lenguaje.

El compilador actual se creó en muy poco tiempo, y, a causa de esto, tiene
múltiples problemas y carencias, y está poco documentado. Esto lo vuelve difícil
de modificar para corregir los problemas y añadir nuevas funcionalidades.
Además, tiene problemas de rendimiento al aumentar el tamaño de los programas.
% TODO: mencionar mejores errores que de los de GNU as?

Debido a esto, se propone crear un nuevo compilador con una arquitectura robusta
que le permita ser flexible para añadir nuevas funcionalidades y corregir
errores, solucionando los problemas del compilador actual. Este nuevo compilador
también mejorará los mensajes de error para ayudar a los estudiantes en su
aprendizaje del lenguaje. Además, añadirá nuevas funcionalidades no soportadas
en el compilador actual, que facilitarán el uso de CREATOR como entorno de
pruebas para nuevas arquitecturas.

\section{Requisitos}\label{sec:requirements}

Esta sección contiene la especificación de los requisitos del sistema a
desarrollar. Para esta especificación de requisitos se han seguido las prácticas
recomendadas por IEEE \parencite{requirementsIEEE}. Estas prácticas indican que
una buena especificación de requisitos debe explicar la funcionalidad del
software, los requisitos de rendimiento, las interfaces externas, otros
atributos, y restricciones de diseño.

\noindent
Además, la especificación de requisitos debe ser:

\begin{itemize}
    \item \textbf{No ambigua:} Los requisitos tienen una única interpretación.
    \item \textbf{Completa:} Incluye todos los requisitos relevantes.
    \item \textbf{Verificable:} Existe un proceso finito y no costoso que permite
    comprobar que el sistema cumple con todos los requisitos.
    \item \textbf{Consistente:} No existe ningún conjunto de requisitos contradictorios
    entre sí.
    \item \textbf{Modificable:} La estructura y estilo de los requisitos permite que
    cualquier cambio necesario se haga de forma fácil, completa, y consistente.
    \item \textbf{Trazable:} El origen de cada requisito es claro, y facilita la
    referencia de cada requisito en otras etapas.
    \item \textbf{Clasificada:} Los requisitos deben estar clasificados según su
    importancia y estabilidad.
\end{itemize}

Para realizar la especificación de requisitos, se parte de los requisitos de
usuario (\subsectionref{user-requirements}), que contienen una especificación
informal de las necesidades del cliente y lo que este espera del producto. A
partir de estos, se crean los requisitos de software
(\subsectionref{software-requirements}). Estos requisitos guían el proceso de
diseño, aportando información sobre las funcionalidades del sistema y cualquier
otra característica adicional.

% ╔════════════════════════════════════════════════════════════════════════════╗
% ║                           Requisitos de usuario                            ║
% ╚════════════════════════════════════════════════════════════════════════════╝

\subsection{Requisitos de usuario}\label{subsec:user-requirements}

Esta sección contiene el listado detallado de los requisitos de usuario. Estos
requisitos explican la funcionalidad principal del sistema y las restricciones
que este debe cumplir para ser aceptado por el cliente.

\noindent
Estos requisitos se pueden dividir en dos tipos:

\begin{itemize}
    \item \textbf{Capacidad:} Representa una funcionalidad que el sistema debe tener.
    \item \textbf{Restricción:} Representa una condición que el sistema debe cumplir.
\end{itemize}

\printureqtemplate

% ───────────────────────────────── Capacidades ───────────────────────────────

\begin{userReq}{CA}{funcionalidades}{pc=h,pd=h,s=c,v=m}
    El sistema debe soportar todas las funcionalidades soportadas en el
    \gls{compiler} actual.
\end{userReq}

\begin{userReq}{CA}{utf8}{pc=m,pd=m,s=c,v=h}
    El sistema debe soportar cadenas de caracteres codificadas en UTF-8
    \parencite{UTF-8}.
\end{userReq}

\begin{userReq}{CA}{escape}{pc=m,pd=m,s=i,v=m}
    El sistema debe soportar secuencias de escape en las cadenas de caracteres y
    caracteres literales.
\end{userReq}

\begin{userReq}{CA}{ISA}{pc=h,pd=h,s=c,v=h}
    El sistema debe permitir especificar la \gls{ISA} a utilizar.
\end{userReq}

\begin{userReq}{CA}{comentarios-tipos}{pc=m,pd=l,s=c,v=h}
    El sistema debe permitir utilizar comentarios de línea y comentarios
    multilínea.
\end{userReq}

\begin{userReq}{CA}{comentarios}{pc=m,pd=l,s=c,v=h}
    El sistema debe permitir especificar la cadena utilizada como prefijo de los
    comentarios de línea.
\end{userReq}

\begin{userReq}{CA}{num-etiquetas}{pc=m,pd=l,s=c,v=h}
    El sistema debe permitir utilizar cualquier cantidad de etiquetas en una
    \gls{sentence}.
\end{userReq}

\begin{userReq}{CA}{referencias}{pc=h,pd=h,s=i,v=m}
    El sistema debe permitir referenciar etiquetas definidas en el
    código \glsdisp{assembly}{ensamblador} después de su uso.
\end{userReq}

\begin{userReq}{CA}{etiquetas-valor}{pc=m,pd=m,s=c,v=h}
    El sistema debe permitir utilizar etiquetas como valores de las \glspl{data directive}.
\end{userReq}

\begin{userReq}{CA}{expresiones}{pc=m,pd=m,s=i,v=h}
    El sistema debe permitir evaluar y utilizar \glspl{expression} con valores
    constantes.
\end{userReq}

\begin{userReq}{CA}{mult-defs-instruccion}{pc=h,pd=m,s=c,v=h}
    El sistema debe permitir definir múltiples \glspl{instruction} con el mismo
    nombre, y utilizar su sintaxis y el tamaño de los argumentos (en bits) para
    seleccionar la correcta durante la \gls{compilation} de un \gls{program}
    \glsdisp{assembly}{ensamblador}.
\end{userReq}

\begin{userReq}{CA}{errores-tipos}{pc=h,pd=h,s=i,v=m}
    El sistema debe detectar errores en el código \glsdisp{assembly}{ensamblador}.
\end{userReq}

\begin{userReq}{CA}{errores}{pc=h,pd=h,s=i,v=l}
    El sistema debe tener buenos mensajes de error.
\end{userReq}

\begin{userReq}{CA}{bigints}{pc=h,pd=m,s=c,v=h}
    El sistema debe permitir utilizar números enteros de cualquier tamaño para
    las direcciones de \gls{memory} y valores de las \glspl{expression}.
\end{userReq}

% ──────────────────────────────── Restricciones ──────────────────────────────

\begin{userReq}{RE}{CREATOR}{pc=h,pd=h,s=c,v=h}
    El sistema se tiene que integrar con el simulador CREATOR.
\end{userReq}

\begin{userReq}{RE}{entornos}{pc=h,pd=m,s=c,v=h}
    El sistema se tiene que poder ejecutar en los entornos soportados
    actualmente por CREATOR (navegadores y Node.js).
\end{userReq}

\begin{userReq}{RE}{FOSS}{pc=h,pd=m,s=c,v=h}
    El sistema tiene que ser \gls{FOSS}.
\end{userReq}

\begin{userReq}{RE}{sintaxis}{pc=h,pd=h,s=i,v=m}
    El sistema tiene que utilizar la sintaxis del \gls{assembler} GNU as
    \parencite{as-manual} para los \glspl{program} \glsdisp{compilation}{compilados}.
\end{userReq}

\begin{userReq}{RE}{velocidad}{pc=m,pd=m,s=vu,v=l}
    El sistema tiene que ser rápido. % TODO: más explicación
\end{userReq}

\FloatBarrier

% ╔════════════════════════════════════════════════════════════════════════════╗
% ║                           Requisitos de software                           ║
% ╚════════════════════════════════════════════════════════════════════════════╝

\subsection{Requisitos de software}\label{subsec:software-requirements}

Esta sección contiene el listado detallado de los requisitos de software. Estos
requisitos se han creado a partir de los requisitos de usuario descritos en la
sección anterior, y definen las características del sistema.

\noindent
Estos requisitos se pueden dividir en dos tipos:

\begin{itemize}
    \item \textbf{Funcional:} Define las funcionalidades y características del
    software.
    \item \textbf{No funcional:} Define una condición que el software debe cumplir.
\end{itemize}

\printsreqtemplate

% ───────────────────────────────── Funcionales ───────────────────────────────

\begin{softwareReq}{FN}{compilar-instrucciones}{pc=h,pd=h,s=c,v=h}
    {CA-funcionalidades}
    El sistema debe permitir \glsdisp{compilation}{compilar} \glspl{instruction}.
\end{softwareReq}

\begin{softwareReq}{FN}{compilar-pseudo}{pc=h,pd=h,s=c,v=h}
    {CA-funcionalidades}
    El sistema debe permitir \glsdisp{compilation}{compilar} \glspl{pseudo-instruction}.
\end{softwareReq}

\begin{softwareReq}{FN}{directivas}{pc=h,pd=h,s=c,v=h}
    {CA-funcionalidades}
    El sistema debe soportar las siguientes \glspl{directive}: cambiar el tipo
    de la sección actual, declarar etiquetas globales, \glspl{data directive}, e
    ignorar directiva.
\end{softwareReq}

\begin{softwareReq}{FN}{secciones}{pc=h,pd=h,s=c,v=h}
    {CA-funcionalidades}
    El sistema debe soportar los siguientes tipos de secciones: datos e
    instrucciones.
\end{softwareReq}

\begin{softwareReq}{FN}{directivas-datos}{pc=h,pd=h,s=c,v=h}
    {CA-funcionalidades}
    El sistema debe soportar las siguientes \glspl{data directive}: reservar
    espacio, cadenas de caracteres, números enteros, números decimales, y
    alineamiento de la \gls{data memory}.
\end{softwareReq}

\begin{softwareReq}{FN}{strings}{pc=h,pd=h,s=c,v=h}
    {CA-funcionalidades}
    El sistema debe soportar \glspl{data directive} de cadenas de
    caracteres terminadas y no terminadas en un \gls{null byte}.
\end{softwareReq}

\begin{softwareReq}{FN}{tamaño-ints}{pc=h,pd=h,s=c,v=h}
    {CA-funcionalidades}
    El sistema debe soportar \glspl{data directive} de números enteros con los
    siguientes tamaños: byte, media \gls{word}, \gls{word}, y doble \gls{word}.
\end{softwareReq}

\begin{softwareReq}{FN}{floats}{pc=h,pd=h,s=c,v=h}
    {CA-funcionalidades}
    El sistema debe soportar \glspl{data directive} de números decimales con el
    formato IEEE 745 \parencite{FloatsIEEE} de precisión simple (binary32) y
    doble (binary64).
\end{softwareReq}

\begin{softwareReq}{FN}{alineamiento}{pc=h,pd=h,s=c,v=h}
    {CA-funcionalidades}
    El sistema debe soportar \glspl{data directive} de alineamiento a bytes y a
    potencias de 2.
\end{softwareReq}

\begin{softwareReq}{FN}{bibliotecas}{pc=h,pd=l,s=c,v=h}
    {CA-funcionalidades}
    El sistema debe soportar \glspl{library}: debe permitir reservar espacio para
    instrucciones de un \gls{program} \glsdisp{compilation}{compilado}
    previamente, y utilizar las etiquetas declaradas como globales definidas en
    este.
\end{softwareReq}

\begin{softwareReq}{FN}{utf8}{pc=m,pd=m,s=c,v=h}
    {CA-utf8}
    El sistema debe soportar cadenas de caracteres codificadas en UTF-8 \parencite{UTF-8}.
\end{softwareReq}

\begin{softwareReq}{FN}{escape}{pc=m,pd=m,s=i,v=h}
    {CA-escape}
    El sistema debe soportar las siguientes secuencias de escape en las cadenas
    de caracteres y caracteres literales: \verb!\\!, \verb!\"!, \verb!\'!,
    \verb!\a! (alerta), \verb!\b! (backspace), \verb!\e! (escape), \verb!\f!
    (salto de página), \verb!\n!, \verb!\r!, \verb!\t!, y \verb!\0!.
\end{softwareReq}

\begin{softwareReq}{FN}{ISA}{pc=h,pd=h,s=c,v=h}
    {CA-ISA}
    El sistema debe permitir especificar la \gls{ISA} a utilizar.
\end{softwareReq}

\begin{softwareReq}{FN}{comentarios-tipos}{pc=m,pd=l,s=c,v=h}
    {CA-comentarios-tipos}
    El sistema debe permitir utilizar comentarios de línea y comentarios
    multilínea.
\end{softwareReq}

\begin{softwareReq}{FN}{comentarios}{pc=m,pd=l,s=c,v=h}
    {CA-comentarios}
    El sistema debe permitir especificar la cadena utilizada como prefijo de los
    comentarios de línea.
\end{softwareReq}

\begin{softwareReq}{FN}{num-etiquetas}{pc=m,pd=l,s=c,v=h}
    {CA-num-etiquetas}
    El sistema debe permitir utilizar cualquier cantidad de etiquetas en una
    \gls{sentence}.
\end{softwareReq}

\begin{softwareReq}{FN}{referencias}{pc=h,pd=h,s=i,v=m}
    {CA-referencias}
    El sistema debe permitir referenciar etiquetas definidas en el
    código \glsdisp{assembly}{ensamblador} después de su uso.
\end{softwareReq}

\begin{softwareReq}{FN}{constantes}{pc=m,pd=m,s=c,v=h}
    {CA-etiquetas-valor, CA-expresiones}
    El sistema debe permitir evaluar \glspl{expression} con los siguientes tipos
    de valores constantes: números enteros, números decimales, caracteres
    literales (utilizando su representación Unicode \parencite{UTF-8}) y etiquetas.
\end{softwareReq}

\begin{softwareReq}{FN}{expr-operadores}{pc=m,pd=m,s=i,v=h}
    {CA-expresiones}
    El sistema debe permitir utilizar paréntesis y los siguientes operadores en
    las \glspl{expression}:
    \begin{itemize}
        \item Unarios: \verb!+!, \verb!-!, y \verb!~! (complemento).
        \item Binarios aritméticos: \verb!+!, \verb!-!, \verb!*!, \verb!/!, y \verb!%!.
        \item Binarios bit a bit: \verb!|! (OR), \verb!&! (AND), y \verb!^! (XOR).
    \end{itemize}
\end{softwareReq}

\begin{softwareReq}{FN}{expr-instruccion}{pc=m,pd=m,s=c,v=h}
    {CA-expresiones}
    El sistema debe permitir utilizar \glspl{expression} como \glspl{immediate}
    de las \glspl{instruction} y \glspl{pseudo-instruction}.
\end{softwareReq}

\begin{softwareReq}{FN}{expr-directiva}{pc=m,pd=m,s=c,v=h}
    {CA-expresiones}
    El sistema debe permitir utilizar \glspl{expression} como valores de las
    \glspl{data directive} de reservar espacio, números enteros y decimales, y
    alineamiento.
\end{softwareReq}

\begin{softwareReq}{FN}{mult-defs-instruccion}{pc=h,pd=m,s=c,v=h}
    {CA-mult-defs-instruccion}
    El sistema debe permitir definir múltiples \glspl{instruction} con el mismo
    nombre, y utilizar su sintaxis y el tamaño de los argumentos (en bits) para
    seleccionar la correcta durante la \gls{compilation} de un \gls{program}
    \glsdisp{assembly}{ensamblador}.
\end{softwareReq}

\begin{softwareReq}{FN}{errores-sintaxis}{pc=h,pd=h,s=c,v=m}
    {CA-errores-tipos}
    El sistema debe detectar errores de sintaxis en el código \glsdisp{assembly}{ensamblador}.
\end{softwareReq}

\begin{softwareReq}{FN}{errores-semantica}{pc=h,pd=h,s=i,v=m}
    {CA-errores-tipos}
    El sistema debe detectar errores semánticos en el código \glsdisp{assembly}{ensamblador}.
\end{softwareReq}

\begin{softwareReq}{FN}{errores-base}{pc=h,pd=h,s=i,v=h}
    {CA-errores}
    Los mensajes de error del sistema deben contener, al menos: posición exacta
    del error (línea y columna), mensaje de error, segmento de código con el
    error resaltado, e información adicional relevante como motivo del error u
    posible solución.
\end{softwareReq}

%\begin{softwareReq}{FN}{errores-extra}{pc=l,pd=l,s=i,v=h}
%    {CA-errores}
%    Los mensajes de error del sistema que se refieran a identificadores no
%    definidos deben contener el identificador definido más similar según la
%    \gls{edit-distance}
%\end{softwareReq}

\begin{softwareReq}{FN}{bigints}{pc=h,pd=m,s=c,v=h}
    {CA-bigints}
    El sistema debe permitir utilizar números enteros de cualquier tamaño para
    las direcciones de \gls{memory} y valores de las \glspl{expression}.
\end{softwareReq}

% ──────────────────────────────── No funcionales ────────────────────────────────

\begin{softwareReq}{NF}{ISA}{pc=h,pd=h,s=c,v=h}
    {RE-CREATOR}
    El formato de definición de la \gls{ISA} debe ser el utilizado
    actualmente en CREATOR, serializado en formato \gls{json}
    \parencite{JSONStandard}. % TODO: add appendix with format description
\end{softwareReq}

\begin{softwareReq}{NF}{def-instrucciones}{pc=h,pd=h,s=c,v=h}
    {RE-CREATOR}
    La definición de la sintaxis de las instrucciones debe estar contenida en la
    \gls{ISA}.
\end{softwareReq}

\begin{softwareReq}{NF}{API}{pc=h,pd=h,s=c,v=h}
    {RE-CREATOR}
    La \gls{API} de entrada y salida del sistema debe ser la utilizada
    actualmente en CREATOR.
\end{softwareReq}

\begin{softwareReq}{NF}{JS}{pc=h,pd=m,s=c,v=h}
    {RE-CREATOR}
    El sistema debe tener una \gls{API} para \gls{JS}.
\end{softwareReq}

\begin{softwareReq}{NF}{chrome}{pc=h,pd=m,s=c,v=h}
    {RE-entornos}
    El sistema se tiene que poder ejecutar en Google Chrome 70+.
\end{softwareReq}

\begin{softwareReq}{NF}{firefox}{pc=h,pd=m,s=c,v=h}
    {RE-entornos}
    El sistema se tiene que poder ejecutar en Mozilla Firefox 60+.
\end{softwareReq}

\begin{softwareReq}{NF}{safari}{pc=h,pd=m,s=c,v=h}
    {RE-entornos}
    El sistema se tiene que poder ejecutar en Safari 10+.
\end{softwareReq}

\begin{softwareReq}{NF}{nodejs}{pc=h,pd=m,s=c,v=h}
    {RE-entornos}
    El sistema se tiene que poder ejecutar en Node.js.
\end{softwareReq}

\begin{softwareReq}{NF}{licencia}{pc=h,pd=m,s=c,v=h}
    {RE-FOSS}
    El sistema tiene que utilizar una licencia \gls{FOSS} según las definiciones
    de la \gls{FSF} \parencite{FreeSoftware} y \gls{OSI} \parencite{OpenSource}.
\end{softwareReq}

\begin{softwareReq}{NF}{codigo-publico}{pc=h,pd=m,s=c,v=h}
    {RE-FOSS}
    El código fuente del sistema tiene ser público.
\end{softwareReq}

\begin{softwareReq}{NF}{sintaxis}{pc=h,pd=h,s=i,v=m}
    {RE-sintaxis}
    El sistema tiene que utilizar la sintaxis del \gls{assembler} GNU as
    \parencite{as-manual} para los \glspl{program} \glsdisp{compilation}{compilados}.
\end{softwareReq}

\begin{softwareReq}{NF}{velocidad}{pc=m,pd=m,s=vu,v=l}
    {RE-velocidad}
    El sistema tiene que ser rápido. % TODO: más explicación
\end{softwareReq}

\FloatBarrier

% ╔════════════════════════════════════════════════════════════════════════════╗
% ║                                Trazabilidad                                ║
% ╚════════════════════════════════════════════════════════════════════════════╝

\subsection{Trazabilidad}\label{subsec:trazability}

La matriz de trazabilidad permite verificar que todos los requisitos de usuario
están cubiertos por al menos un requisito de software. Como se puede ver, todos
los requisitos de capacidad están cubiertos por los requisitos funcionales
(\tableref{trazability-ca-f}), y todos los requisitos de restricción están
cubiertos por los requisitos no funcionales (\tableref{trazability-re-nf})

\traceabilityTable{trazability-ca-f}{^RS\37-FN}{^RU\37-CA}
    {Trazabilidad entre los requisitos de capacidad y los requisitos funcionales}

\traceabilityTable{trazability-re-nf}{^RS\37-NF}{^RU\37-RE}
    {Trazabilidad entre los requisitos de capacidad y los requisitos funcionales}

\FloatBarrier

\section{Casos de uso}\label{sec:usecases}

El modelo de casos de uso UML \parencite{UMLSpec}, mostrado en la
\figureref{use_cases}, representa las acciones que el usuario puede realizar con
el sistema. Cabe destacar que el usuario del sistema será el simulador CREATOR,
ya que el sistema a desarrollar será un componente utilizado internamente por
CREATOR para llevar a cabo su funcionalidad.

\drawiosvgfigure[0.55]{use_cases}{Modelo de casos de uso}

\printuctemplate

\begin{useCase}{cargar}
    {Cargar \gls{ISA}} % Nombre
    {CREATOR} % Usuario
    {Cargar la \gls{ISA} a utilizar durante la \gls{compilation} de un código \glsdisp{assembly}{ensamblador}.} % Description
    {CREATOR debe tener la \gls{ISA} correctamente definida en una cadena de caracteres.} % Pre-condición
    {La \gls{ISA} está cargada y disponible para realizar la compilación.} % Post-condición
    \begin{enumerate}[leftmargin=*, topsep=0pt, noitemsep]
        \item CREATOR entrega una cadena de caracteres con la
        \gls{ISA} a la \gls{API} de cargar \gls{ISA}.
    \end{enumerate}
\end{useCase}

\begin{useCase}{compilar}
    {\glsdisp{compilation}{Compilar} \gls{program}} % Nombre
    {CREATOR} % Usuario
    {\glsdisp{compilation}{Compilar} un \gls{program} para su posterior ejecución en CREATOR.} % Description
    {La \gls{ISA} debe estar cargada en el \gls{compiler}.} % Pre-condición
    {El \gls{program} está compilado y disponible para su ejecución en CREATOR.} % Post-condición
    \begin{enumerate}[leftmargin=*, topsep=0pt, noitemsep]
        \item CREATOR entrega una cadena de caracteres con el \gls{program}
        a la \gls{API} de \glsdisp{compilation}{compilar} \gls{program}.
    \end{enumerate}
\end{useCase}
