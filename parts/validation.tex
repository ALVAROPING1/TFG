\chapter{Validación y verificación}\label{chap:validation}

\printtesttemplate

\section{Verificación}

\begin{testCase}{VET}{ISA}
    {El sistema está instalado.} % Precondición
    {Se obtiene el código compilado correspondiente al programa introducido.} % Postcondición
    {Verificar que el sistema puede compilar \glspl{program} en una \gls{ISA}.} % Descripción
    {OK} % Evaluación
    {FN-compilar-instrucciones, FN-directivas, FN-secciones, FN-ISA,
    FN-num-etiquetas, FN-referencias, NF-ISA, NF-def-instrucciones, NF-sintaxis} % Origen
    \begin{enumerate}[leftmargin=*, topsep=0pt, noitemsep] % Procedimiento
        \item Implementar una \gls{ISA} básica con diferentes \glspl{register},
        \glspl{instruction}, y \glspl{directive}.
        \item Escribir un \gls{program} que utilice todos los \glspl{register},
        \glspl{instruction}, y \glspl{directive} definidas en la \gls{ISA},
        múltiples etiquetas en una misma \gls{instruction} y \gls{data directive},
        y etiquetas definidas después de su uso.
        \item Cargar la \gls{ISA} en el compilador.
        \item Compilar el \gls{program}.
    \end{enumerate}
\end{testCase}

\begin{testCase}{VET}{pseudo}
    {El sistema está instalado.} % Precondición
    {Se obtiene el código compilado con las secuencias de  \glspl{instruction}
    correspondientes a las \glspl{pseudo-instruction} introducidas.} % Postcondición
    {Verificar que el sistema puede compilar \glspl{pseudo-instruction}.} % Descripción
    {OK} % Evaluación
    {FN-compilar-instrucciones, FN-compilar-pseudo} % Origen
    \begin{enumerate}[leftmargin=*, topsep=0pt, noitemsep] % Procedimiento
        \item Implementar una \gls{ISA} básica con diferentes
        \glspl{instruction} y \glspl{pseudo-instruction}.
        \item Escribir un \gls{program} que utilice las \glspl{pseudo-instruction},
        definidas en la \gls{ISA}.
        \item Cargar la \gls{ISA} en el compilador.
        \item Compilar el \gls{program}.
    \end{enumerate}
\end{testCase}

\begin{testCase}{VET}{directivas-datos}
    {El sistema está instalado y se tiene una \gls{ISA} cargada con todos los
    tipos de directivas de datos soportadas.} % Precondición
    {Se obtiene el código compilado con los valores de las \gls{data directive} introducidas.} % Postcondición
    {Verificar que el sistema permite el uso de todas las \glspl{data directive} soportadas.} % Descripción
    {OK} % Evaluación
    {FN-directivas, FN-directivas-datos, FN-strings, FN-tamaño-ints, FN-floats, FN-alineamiento} % Origen
    \begin{enumerate}[leftmargin=*, topsep=0pt, noitemsep] % Procedimiento
        \item Escribir un \gls{program} que utilice todos los tipos de
        \glspl{data directive} soportadas:
        \begin{itemize}
            \item Cadenas de caracteres terminadas y no terminadas en un
            \gls{null byte}.
            \item Números enteros de todos los tamaños soportados.
            \item Números decimales de precisión simple y doble.
            \item Reservar espacio.
            \item Alinear la memoria de datos a una potencia de 2 y a un tamaño en bytes.
        \end{itemize}
        \item Compilar el \gls{program}.
    \end{enumerate}
\end{testCase}

\begin{testCase}{VET}{bibliotecas}
    {El sistema está instalado y se tiene una \gls{ISA} cargada.} % Precondición
    {Se obtiene el código compilado correspondiente al \gls{program} introducido.} % Postcondición
    {Verificar que el sistema permite el uso \glspl{library}.} % Descripción
    {OK} % Evaluación
    {FN-directivas, FN-bibliotecas} % Origen
    \begin{enumerate}[leftmargin=*, topsep=0pt, noitemsep] % Procedimiento
        \item Escribir un \gls{program} que declare algunas de sus etiquetas
        como globales.
        \item Compilar el \gls{program} y guardarlo como una \gls{library}.
        \item Escribir un nuevo \gls{program} que utilice las etiquetas
        declaradas como globales en el programa original.
        \item Compilar el nuevo \gls{program} añadiendo el original como una
        \gls{library}.
    \end{enumerate}
\end{testCase}

\begin{testCase}{VET}{strings}
    {El sistema está instalado y se tiene una \gls{ISA} cargada.} % Precondición
    {Se obtiene el código compilado en la codificación UTF-8 \parencite{UTF-8}
    correspondiente a la cadena de caracteres introducida.} % Postcondición
    {Verificar que el sistema permite el uso de cadenas de caracteres con
    secuencias de escape codificadas en UTF-8 \parencite{UTF-8}.} % Descripción
    {OK} % Evaluación
    {FN-utf8, FN-escape} % Origen
    \begin{enumerate}[leftmargin=*, topsep=0pt, noitemsep] % Procedimiento
        \item Escribir un \gls{program} con una cadena que contenga todas las
        secuencias de escape soportadas y caracteres no ASCII.
        \item Compilar el \gls{program}.
    \end{enumerate}
\end{testCase}

\begin{testCase}{VET}{comentarios}
    {El sistema está instalado y se tiene una \gls{ISA} cargada.} % Precondición
    {Se obtiene el código compilado correspondiente al programa, ignorando los
    comentarios introducidos.} % Postcondición
    {Verificar que el sistema permite el uso comentarios y la selección del
    prefijo de los comentarios de línea.} % Descripción
    {OK} % Evaluación
    {FN-comentarios-tipos, FN-comentarios} % Origen
    \begin{enumerate}[leftmargin=*, topsep=0pt, noitemsep] % Procedimiento
        \item Escribir un \gls{program} con comentarios de línea y multilínea.
        \item Compilar el \gls{program} y verificar que los comentarios son
        ignorados.
        \item Modificar el prefijo de los comentarios de línea en la definición
        de la \gls{ISA}, y cargar la nueva versión.
        \item Modificar el prefijo de los comentarios de línea en el
        \gls{program}.
        \item Compilar el \gls{program}.
    \end{enumerate}
\end{testCase}

\begin{testCase}{VET}{expresiones-int}
    {El sistema está instalado y se tiene una \gls{ISA} cargada.} % Precondición
    {Se obtiene el código compilado con los valores de las \glspl{expression} introducidas.} % Postcondición
    {Verificar que el sistema permite el uso \glspl{expression} de números enteros.} % Descripción
    {OK} % Evaluación
    {FN-constantes, FN-expr-operadores, FN-expr-instruccion, FN-expr-directiva} % Origen
    \begin{enumerate}[leftmargin=*, topsep=0pt, noitemsep] % Procedimiento
        \item Escribir un \gls{program} que utilice \gls{expression} de números
        enteros, caracteres literales, y etiquetas, y todos los operadores
        soportados, como \glspl{immediate} de \glspl{instruction} y valores de
        \gls{data directive} que acepten números enteros como argumentos.
        \item Compilar el \gls{program}.
    \end{enumerate}
\end{testCase}

\begin{testCase}{VET}{mult-defs-instruccion}
    {El sistema está instalado.} % Precondición
    {Se obtiene el código compilado correspondiente a las \glspl{instruction} introducidas.} % Postcondición
    {Verificar que el sistema permite definir múltiples \glspl{instruction} con
    el mismo nombre, y utilizar su sintaxis y el tamaño de sus campos para
    seleccionar la correcta durante la compilación.} % Descripción
    {OK} % Evaluación
    {FN-mult-defs-instruccion} % Origen
    \begin{enumerate}[leftmargin=*, topsep=0pt, noitemsep] % Procedimiento
        \item Implementar una \gls{ISA} con múltiples \glspl{instruction} con el
        mismo nombre pero diferente sintaxis y/o tamaño de sus argumentos.
        \item Escribir un \gls{program} que todas las definiciones de la
        \gls{instruction} creada.
        \item Cargar la \gls{ISA} en el compilador.
        \item Compilar el \gls{program}.
    \end{enumerate}
\end{testCase}

\begin{testCase}{VET}{errores-sintaxis}
    {El sistema está instalado y se tiene una \gls{ISA} cargada.} % Precondición
    {Se obtiene un mensaje de error de sintaxis con toda la información pedida en el \sreqref{FN-errores-base}.} % Postcondición
    {Verificar que el sistema detecta errores de sintaxis y genera buenos mensajes de error.} % Descripción
    {OK} % Evaluación
    {FN-errores-sintaxis, FN-errores-base} % Origen
    \begin{enumerate}[leftmargin=*, topsep=0pt, noitemsep] % Procedimiento
        \item Escribir un \gls{program} con un error de sintaxis.
        \item Compilar el \gls{program}.
    \end{enumerate}
\end{testCase}

\begin{testCase}{VET}{errores-semantica}
    {El sistema está instalado y se tiene una \gls{ISA} cargada.} % Precondición
    {Se obtiene un mensaje de error semántico con toda la información pedida en el \sreqref{FN-errores-base}.} % Postcondición
    {Verificar que el sistema detecta errores semánticos y genera buenos mensajes de error.} % Descripción
    {OK} % Evaluación
    {FN-errores-semantica, FN-errores-base} % Origen
    \begin{enumerate}[leftmargin=*, topsep=0pt, noitemsep] % Procedimiento
        \item Escribir un \gls{program} con un error semántico.
        \item Compilar el \gls{program}.
    \end{enumerate}
\end{testCase}

\begin{testCase}{VET}{bigints}
    {El sistema está instalado.} % Precondición
    {Se obtiene el código compilado con el valor introducido.} % Postcondición
    {Verificar que el sistema detecta errores semánticos y genera buenos mensajes de error.} % Descripción
    {OK} % Evaluación
    {FN-bigints} % Origen
    \begin{enumerate}[leftmargin=*, topsep=0pt, noitemsep] % Procedimiento
        \item Implementar una \gls{ISA} con una \gls{data directive} de
        enteros con un tamaño máximo arbitrariamente grande.
        \item Escribir un \gls{program} que utilice la \gls{data directive} de
        enteros con un número entero cercano al valor máximo definido.
        \item Cargar la \gls{ISA} en el compilador.
        \item Compilar el \gls{program}.
    \end{enumerate}
\end{testCase}

\begin{testCase}{VET}{API}
    {El código del sistema está descargado} % Precondición
    {El sistema utiliza la \gls{API} de entrada y salida en \gls{JS}
    implementada en el \gls{compiler} actual.} % Postcondición
    {Verificar que el sistema utiliza la \gls{API} implementada en el \gls{compiler} actual.} % Descripción
    {OK} % Evaluación
    {NF-API, NF-JS} % Origen
    \begin{enumerate}[leftmargin=*, topsep=0pt, noitemsep] % Procedimiento
        \item Acceder a la definición de la interfaz de \gls{JS} del \gls{compiler}.
        \item Acceder a la definición de la interfaz del \gls{compiler} actual.
    \end{enumerate}
\end{testCase}

\begin{testCase}{VET}{plataformas}
    {El sistema está instalado.} % Precondición
    {El sistema funciona correctamente en todas las plataformas probadas.} % Postcondición
    {Verificar que el sistema funciona en todas las plataformas soportadas por
    el \gls{compiler} actual.} % Descripción
    {OK} % Evaluación
    {NF-chrome, NF-firefox, NF-safari, NF-nodejs} % Origen
    \begin{enumerate}[leftmargin=*, topsep=0pt, noitemsep] % Procedimiento
        \item Acceder el sistema desde Google Chrome 70, Mozilla Firefox 70,
        Safari 10, y Node.js.
        \item Utilizar el sistema en esas plataformas.
    \end{enumerate}
\end{testCase}

\begin{testCase}{VET}{FOSS}
    {\NA} % Precondición
    {El sistema cumple con los requisitos para ser \gls{FOSS}} % Postcondición
    {Verificar que el sistema es \gls{FOSS}.} % Descripción
    {OK} % Evaluación
    {NF-licencia, NF-codigo-publico} % Origen
    \begin{enumerate}[leftmargin=*, topsep=0pt, noitemsep] % Procedimiento
        \item Acceder al repositorio con el código fuente del sistema.
        \item Verificar que el código fuente es público y se distribuye con una licencia \gls{FOSS}.
    \end{enumerate}
\end{testCase}

\begin{testCase}{VET}{velocidad}
    {El sistema está instalado y se tiene una \gls{ISA} cargada} % Precondición
    {El sistema cumple con los requisitos de velocidad} % Postcondición
    {Verificar que el sistema es rápido.} % Descripción
    {OK} % Evaluación
    {NF-velocidad} % Origen
    \begin{enumerate}[leftmargin=*, topsep=0pt, noitemsep] % Procedimiento
        \item Escribir un \gls{program} complejo.
        \item Compilar el \gls{program}.
    \end{enumerate}
\end{testCase}

\begin{landscape}
    \traceabilityTable[p]{traceability-rs-vet}{^VET}{^RS\37-FN}
        {Trazabilidad entre los requisitos de software funcionales y las pruebas de verificación}
    \traceabilityTable[p]{traceability-rs-vet}{^VET}{^RS\37-NF}
        {Trazabilidad entre los requisitos de software no funcionales y las pruebas de verificación}
\end{landscape}

\section{Validación}

