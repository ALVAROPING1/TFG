\chapter{Conclusiones y trabajos futuros}\label{chap:conclusions}

En este capítulo se presentarán las conclusiones obtenidas con la realización
del proyecto (\sectionref{proyect-conclusions}), estudiando los objetivos
propuestos en el \chapterref{introduction}, y las conclusiones personales
(\sectionref{personal-conclusions}), indicando los conocimientos adquiridos y
las asignaturas relacionadas con el proyecto. Tras esto, la
\sectionref{additional-contributions} explicará contribuciones adicionales
realizadas durante el desarrollo del proyecto que no están directamente
relacionadas con el mismo. Finalmente, la \sectionref{future-work} indicará
trabajos futuros que se podrían realizar para mejorar el sistema desarrollado.

\section{Conclusiones del proyecto}\label{sec:proyect-conclusions}

En este documento se ha descrito el análisis, diseño, e implementación de un
\gls{compiler} de \gls{assembly} genérico. El sistema desarrollado es
\gls{FOSS}, y está centrado en tener muchas funcionalidades y generar buenos
mensajes de error que ayuden a los usuarios aprender el \gls{assembly}.

Como se explicó en la \sectionref{objectives}, el objetivo principal del
proyecto era crear un \gls{compiler} capaz de reemplazar al actualmente
utilizado por CREATOR, lo cual se ha logrado. Además, también se han logrado
todos los objetivos secundarios:

\begin{itemize}
    \item \textbf{O1:} El sistema permite definir las \glspl{instruction}
    permitidas en la \gls{ISA}, y utilizar estas en los \glspl{program}.
    \item \textbf{O2:} El sistema permite \glsdisp{compilation}{compilar}
    \glspl{program} según la definición de una \gls{ISA}
    \item \textbf{O3:} Los mensajes de error generados por el sistema incluyen
    toda la información relevante para comprender la causa del problema y cómo
    solucionarlo, incluyendo notas con información adicional relevante para el
    error y mensajes de ayuda con posibles soluciones.
    \item \textbf{O4:} El sistema es compatible con todas las definiciones de
    \glspl{ISA} creadas previamente para CREATOR.
    \item \textbf{O5:} El sistema tiene soporte para nuevas funcionalidades
    avanzadas como el uso de \gls{expression} aritméticas y etiquetas como
    valores. Esto permite, por ejemplo, definir los vectores de interrupciones
    utilizados en la asignatura de \textit{Sistemas Operativos}.
\end{itemize}

La principal dificultad encontrada durante el desarrollo del sistema fue el
analizador semántico y su gestión de las \glspl{forward-reference}. Como se
explicó en la \subsectionref{compiler}, se consideraron múltiples opciones para
realizar esto antes de encontrar una buena solución del problema. Además, como
se puede ver en el diagrama de Gantt (\figureref{gantt}), esto requirió mucho
tiempo. Un mayor conocimiento sobre \glspl{assembler} podría haber reducido este
tiempo.

\section{Conclusiones personales}\label{sec:personal-conclusions}

La realización de este proyecto me ha permitido aplicar los conocimientos
adquiridos durante el grado, además de aprender nuevas herramientas y
tecnologías que no había tenido la oportunidad de utilizar hasta ahora.

La principal fuente de información para este proyecto ha sido la asignatura
\textit{Procesadores del Lenguaje}, ya que me enseñó la estructura, componentes,
y funcionamiento de un \gls{compiler}. Otras asignaturas que han sido
fundamentales para su desarrollo han sido \textit{Estructura de Computadores} y
\textit{Arquitectura de Computadores}, en las que se explica el funcionamiento
de un \gls{computer} y sobre el \gls{assembly}. Estas asignaturas son de las que
más me han gustado de la carrera. Otras asignaturas que me ayudaron para este
proyecto fueron \textit{Programación}, \textit{Estructura de Datos y
Algoritmos}, y \textit{Teoría de Autómatas y Lenguajes Formales}.

He aprendido mucho con la realización de este proyecto, tanto con la
implementación del \textit{software} como escribiendo este documento. Aunque ya
conocía Rust antes de empezar y había escrito algunos programas pequeños, no
tenía demasiada experiencia realizando \glspl{program} más grandes y complejos,
y tampoco tenía ninguna experiencia utilizando \gls{wasm}. Escribí este
documento en \LaTeX, y, aunque ya tenía cierta experiencia utilizando esta
herramienta para otros documentos, no había utilizado ninguna de sus
funcionalidades avanzadas. Escribir este documento me ha permitido aprender
mucho más de esta herramienta, y he podido utilizar muchas de sus
funcionalidades avanzadas.

\section{Contribuciones adicionales}\label{sec:additional-contributions}

Como ya se ha mencionado, este documento está escrito en
\LaTeX~\parencite{latex}. Su código fuente se encuentra disponible en
\myreportrepo. Este código fuente utiliza una plantilla de \LaTeX~desarrollada
por Luis Daniel Casais Mezquida \parencite{UC3Mtemplate}. Esta plantilla
contiene una clase de \LaTeX~que define el formato del documento, siguiendo la
normativa de la universidad \parencite{report-guidelines}.

Durante la escritura de este documento, se encontraron y corrigieron varios
fallos en esta plantilla, particularmente en su uso para documentos en Español.
Estas aportaciones se añadieron a la plantilla para permitir su uso por otros
estudiantes. Además, el código fuente de esta memoria puede servir como un
ejemplo de uso de la plantilla, para ayudar a futuros estudiantes que quieran
utilizarla.

Este documento también utiliza el paquete SRS, originalmente desarrollado por
Javier López Gómez y modificado por Luis Daniel Casais Mezquida
\parencite{srs-package}. Este paquete automatiza el proceso de generación de
tablas para el proceso de ingeniería de \textit{software}, particularmente
tablas de requisitos, casos de uso, y componentes, y matrices de trazabilidad
entre estos elementos. Durante la realización de este documento, se modificó
este paquete para incluir nuevas funcionalidades: uso de identificadores
nominales para evitar problemas de referencias al reordenarlos, gestión de
errores para identificadores desconocidos o repetidos, y posibilidad de incluir
múltiples orígenes en todos los elementos. Estos cambios se contribuyeron al
paquete original para permitir su uso por otros estudiantes de ingeniería
informática.

\section{Trabajos futuros}\label{sec:future-work}

Este proyecto implementa un \gls{compiler} de \glsdisp{assembly}{ensamblador},
pero se podría mejorar de diversas formas:

\begin{itemize}
    \item Se podría añadir soporte para \glspl{macro}.
    \item Se podría añadir soporte para la definición de símbolos con valores
    arbitrarios (similar a la directiva \verb!.set! de \gls{GAS}).
    \item La mayoría de \glspl{compiler} implementan estrategias de recuperación
    de errores para permitir detectar la mayor cantidad posible de errores en
    cada \gls{compilation}. Se podrían añadir estas estrategias para poder
    mostrar al usuario la mayor cantidad posible de problemas en su código.
    \item El sistema actualmente soporta operadores aritméticos y bit a bit
    básicos, pero se podrían añadir más operadores como desplazamientos,
    comparaciones, o booleanos.
\end{itemize}

\noindent
Además, también se podrían mejorar algunos aspectos de la implementación del
sistema:

\begin{itemize}
    \item Como se mencionó en la \subsectionref{parser}, crear un analizador
    sintáctico descendente recursivo a mano podría permitir una mayor velocidad.
    \item Por compatibilidad con definiciones antiguas de \glspl{ISA}, el
    sistema utiliza una expansión de \gls{pseudo-instruction} muy similar a la
    implementada por el \gls{compiler} antiguo, y esto evita el uso de
    \glspl{forward-reference} en ciertos casos. Si se acepta modificar estas
    definiciones, se podría rehacer este sistema para corregir este fallo.
\end{itemize}
