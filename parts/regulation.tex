\chapter{Marco regulatorio}\label{chap:regulation}

En este capítulo se expondrá las restricciones legales que se aplican al
proyecto, y está dividido en tres secciones. La primera
(\sectionref{legislation}), explicará las leyes y regulaciones aplicables al
proyecto. La segunda (\sectionref{standards}), hará un resumen de los estándares
técnicos utilizados en el proyecto. Por último, la tercera
(\sectionref{licenses}), indicará las licencias bajo las que se distribuye el
sistema y sus componentes.

\section{Legislación}\label{sec:legislation}

El sistema se ejecuta localmente, y, aunque se puede utilizar en un entorno web,
se ejecuta exclusivamente en el navegador cliente. Además, el sistema no utiliza
ni transmite datos personales, por lo que no se aplica ninguna ley de protección
de datos como la Ley Orgánica de Protección de Datos Personales y Garantía de
los Derechos Digitales (en adelante, ``LOPDGDD''). Aunque el código
\glsdisp{assembly}{ensamblador} que procesa el sistema se podría considerar
propiedad intelectual, este no se almacena ni se envía, por lo que no se aplica
ninguna regulación. Además, no hay ningún riesgo involucrado en la ejecución del
sistema, por lo que no hay ninguna otra regulación que se aplique al sistema.

\section{Estándares técnicos}\label{sec:standards}

\noindent
El sistema utiliza usa los siguientes estándares técnicos:

\begin{itemize}
    \item ISO/IEC 21778:2017 \parencite{JSONStandard}, el estándar del formato
    JSON utilizado para la definición de las \glspl{ISA}.
    \item ISO/IEC 10646:2020 \parencite{UTF-8}, el estándar de la codificación
    UTF-8 usada en las cadenas de caracteres.
    \item La especificación de \gls{wasm} \parencite{wasm-spec}, utilizado para
    ejecutar el compilador en un entorno web.
\end{itemize}

\section{Licencias}\label{sec:licenses}

El sistema utiliza y redistribuye múltiples bibliotecas externas:

\begin{itemize}
    \item \libref{serde}, \libref{serde\string_json}, \libref{ariadne},
    \libref{chumsky}, \libref{regex}, \libref{once\string_cell}
    \libref{num-bigint}, \libref{num-traits}, \libref{wasm-bindgen}, y
    \libref{ansi-to-html} utilizan la licencia MIT \parencite{MIT}, que permite
    utilizar, copiar, modificar, publicar, y redistribuir el código sin
    restricciones.
    \item \libref{self\string_cell} utiliza la licencia Apache 2.0
    \parencite{apache2}, que permite utilizar, copiar, modificar, publicar, y redistribuir el código sin
    restricciones.
\end{itemize}

El sistema desarrollado utiliza la licencia GNU Lesser General Public License
version 3 \parencite{lgpl}, ya que permite el uso, modificación, y
redistribución del código de forma libre con la única restricción de que las
modificaciones se tienen que publicar, asegurando que el sistema continua siendo
abierto.
