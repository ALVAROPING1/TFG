\chapter{Implementación y Despliegue}\label{chap:implementation}

Este capítulo describirá la implementación y despliegue del sistema y está
dividido en dos secciones. La primera (\sectionref{implementation}), hará un
resumen de las decisiones tomadas durante la implementación del sistema, junto
con una descripción de la estructura de ficheros creada para organizar el código
fuente. La segunda (\sectionref{deployment}), describirá los requisitos y pasos
necesarios para desplegar el sistema.

\section{Implementación}\label{sec:implementation}

Como se indicó en la \subsectionref{language}, el sistema se ha desarrollado en
Rust \parencite{Rust}, utilizando la edición de 2021. Se ha utilizado Wasm-pack
\parencite{Wasm-pack} para la generación de \glspl{binding} de \gls{JS}.

El sistema se ha implementado como se indicó en la \sectionref{architecture}.
Cada subsistema se ha implementado en un módulo, con sus componentes almacenados
en submódulos. Cabe destacar que el componente principal de cada subsistema se
ha implementado en el módulo principal del subsistema. Además, se han añadido
submódulos adicionales al analizador semántico para la gestión de diversas
funcionalidades necesarias: campos de bits, etiquetas y \gls{symbol table},
enteros de tamaño fijo, y gestión de secciones de memoria. También se ha añadido
un módulo para la implementación de la \gls{API} de \gls{JS}. Dentro de cada
módulo y submódulo también se incluyen pruebas unitarias que verifican su
correcta funcionalidad.

\noindent
Para la implementación del sistema se han utilizado múltiples bibliotecas
externas:

\newcommand{\libref}[1]{\textit{#1} \parencite{#1}}

\begin{itemize}
    \item Para la deserialización de \gls{json} se han utilizado \libref{serde}
    y \libref{serde\string_json}.
    \item Para el renderizado de los errores se ha utilizado \libref{ariadne}.
    \item Como se indicó en la \subsectionref{parser}, para la realización del
    analizador sintáctico se ha utilizado \libref{chumsky}.
    \item Para el uso de \glspl{regex} se ha utilizado \libref{regex} y
    \libref{once\string_cell}.
    \item Para el uso de números enteros de tamaño arbitrario se ha utilizado
    \libref{num-bigint} y \libref{num-traits}.
    \item Para la generación de \glspl{binding} para \gls{JS} se ha utilizado
    \libref{wasm-bindgen}, y \libref{self\string_cell}.
    \item Para el soporte de renderización de errores con formato HTML se ha
    utilizado \libref{ansi-to-html}.
\end{itemize}

La \figureref{file-structure} muestra la estructura de ficheros del proyecto,
relacionando los ficheros con los componentes descritos. Cabe destacar que los
módulos y submódulos se han organizado siguiendo el estándar moderno de Rust,
según el cual un módulo se implementa en un fichero, y sus submódulos en ficheros
dentro de un directorio con el nombre del módulo padre \parencite{Rust}.

\newcommand{\componentref}[1]{\hyperref[req:#1]{#1}}

\begin{figure}[htb]
  \ffigbox[\FBwidth]
    {%
      \caption{Estructura de ficheros del proyecto}
      \label{fig:file-structure}
    }
    {
      \begin{tcolorbox}
        \dirtree{%
          .1 /.
            .2 js\_example/\DTcomment{Ejemplos de uso}.
              .3 compiler.mjs\DTcomment{Uso compilador}.
              .3 index.html\DTcomment{Página navegador}.
              .3 node.js\DTcomment{Carga en Node.js}.
              .3 web.js\DTcomment{Carga en navegador}.
            .2 src/.
              .3 architecture/\DTcomment{Deserialización de la \gls{ISA}}.
                .4 json.rs\DTcomment{Deserialización \gls{json}}.
                .4 utils.rs\DTcomment{Utilidades de deserialización \gls{json}}.
              .3 compiler/\DTcomment{Subsistema analizador semántico}.
                .4 bit\_field.rs\DTcomment{Campo de bits}.
                .4 error.rs\DTcomment{Gestión errores}.
                .4 integer.rs\DTcomment{Enteros}.
                .4 label.rs\DTcomment{Etiquetas y \gls{symbol table}}.
                .4 pseudoinstruction.rs\DTcomment{\componentref{Evaluador de pseudo-instrucciones}}.
                .4 section.rs\DTcomment{Gestor de segmentos de \gls{memory}}.
              .3 parser/\DTcomment{Subsistema analizador sintáctico}.
                .4 error.rs\DTcomment{Gestión errores}.
                .4 expression.rs\DTcomment{\componentref{Parser de expressiones}}.
                .4 instruction.rs\DTcomment{\componentref{Parser de instrucciones}}.
                .4 lexer.rs\DTcomment{\componentref{Lexer}}.
              .3 architecture.rs\DTcomment{\componentref{Gestor de la arquitectura}}.
              .3 compiler.rs\DTcomment{\componentref{Traductor}}.
              .3 js.rs\DTcomment{\gls{API} \gls{JS}}.
              .3 lib.rs\DTcomment{Biblioteca del \gls{compiler}}.
              .3 parser.rs\DTcomment{\componentref{Parser}}.
              .3 span.rs\DTcomment{Región de código}.
            .2 tests/.
              .3 architecture.json\DTcomment{Arquitectura de prueba}.
            .2 Cargo.lock\DTcomment{Versiones de dependencias}.
            .2 Cargo.toml\DTcomment{Metadatos biblioteca}.
            .2 LICENSE.
            .2 README.md.
            .2 build.sh\DTcomment{Compilación biblioteca}.
        }
      \end{tcolorbox}
    }
\end{figure}

\let\componentref\undefined

\noindent
El código fuente está disponible en \myrepo.

\FloatBarrier

\section{Despliegue}\label{sec:deployment}

\noindent
Las especificaciones técnicas recomendadas para que un usuario tenga una buena
experiencia utilizando la herramienta son las siguientes:

\begin{itemize}
    \item \textbf{Sistema Operativo (SO):} Ubuntu 22.04 LTS (Linux) / Windows 11.
    \item \textbf{Procesador:} Intel\textsuperscript{\textregistered} Core\textsuperscript{TM} i3 CPU 7100 @3.9GHz o superior.
    \item \textbf{\Gls{memory} RAM:} 2GB o superior.
    \item \textbf{Almacenamiento:} 1GB o superior.
    \item \textbf{Red:} se necesita una conexión de red para descargar las
    dependencias durante la compilación del sistema.
    \item \textbf{Software:}
    \begin{itemize}
        \item Los navegadores recomendados para utilizar el sistema son Google
        Chrome 70+, Mozilla Firefox 60+, y Safari 12+.
        \item Para compilar el sistema, es necesario tener instalado el compilador
        de Rust y \verb!cargo! \parencite{rust-toolchain}, y Wasm-pack
        \parencite{Wasm-pack}.
    \end{itemize}
\end{itemize}

\noindent
Para compilar el sistema, es necesario seguir los siguientes pasos:

\begin{enumerate}
    \item Descargar y descomprimir el código fuente desde el repositorio.
    \item Compilar el código fuente con Wasm-pack \parencite{Wasm-pack}. Se
    incluye un script con el código fuente para simplificar esta tarea,
    que se puede ejecutar con el comando \verb!./build.sh!.
    \item Incluir el sistema en una aplicación web o Node.js. Junto con el
    código fuente, dentro del directorio \verb!js_example!, se incluye un
    ejemplo de cómo se puede realizar esta tarea.
\end{enumerate}

Además, el sistema se ha integrado en CREATOR, que está disponible en la
dirección \url{https://creatorsim.github.io/}. Esto permite el uso del sistema
sin necesidad de realizar el proceso anterior.
